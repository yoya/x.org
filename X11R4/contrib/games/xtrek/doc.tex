% doc.tex
% this contains some doc about xtrek
% much of it is taken directly from the original xtrek doc

\title{Xtrek version 5.4}

\author{Daniel Lovinger dl2n+@andrew.cmu.edu}
\author{Jon Bennett jcrb@cs.cmu.edu}
\author{Mike Bolotski mikeb@salmon.ee.ubc.ca}
\author{David Gagne daveg@salmon.ee.ubc.ca}

\date{September 29, 1989}

\parindent 0in
\parskip 0.10in 
\textwidth 6.5in
\oddsidemargin 0in
\evensidemargin 0in
\textheight 8.175in
\topmargin -0.25in

\begin{document}

\maketitle

\tableofcontents \newpage

\section{Introduction}

Xtrek is one of the all time classic games for X Windows. This release of the game at
version 5.4 represents almost five straight months of work by various people at 
Carnegie Mellon University and the University of British Columbia, and in our humble
opinion constitutes a major improvement over the previous version 4.0. For those familiar
with version 4.0, much of the game remains on the surface the same : you still fly a ship
'round the galaxy blowing away your friends and conquering planets. However, where the old
xtrek was cast in stone (or executable), this version is configurable in almost every respect
from the power of your torpedoes to the number of configuration of the galaxy. A whole new
universe is only a few hours of configuration file hacking away ...

The basic object of the game is to conquer sixteen (by default - as with almost everything, 
this is configurable) planets for your empire. Of course, everyone else is also, and that is where
all of the fun comes in.

\section{Getting Started}

Xtrek is split up into two seperate programs, xtrek and xtrekd. Xtrekd is the main server process for the game
and accepts as a command line argument the name of a configuration file for the game. For instance

xtrekd cmu.config

will start a game with the contents of the cmu.config file. The daemon first tries to open the config file
as an absolute pathname, and then attempts to load the config file from the xtrek library directory (specified
on compile time). If no config file is specified, ``default.config'' is read from the library. The daemon will
print out status information during the course of the game, including the number of players each time a death
or entrance occurs, and notifies you whenever a robot enters for a given empire.

\subsection{Entering the Game}

Once xtrekd is running, players can connect to it by running xtrek with the machine name the daemon is running
on as the argument on the command line. For instance, assume that a player has a running xtrekd on ligonier.andrew.cmu.edu.
Then

xtrek ligonier.andrew.cmu.edu

would send a request to the daemon to open an xtrek window on the local machine. Various errors can occur at this point,
most of which have to do with access protections on you X server - you must first xhost to the machine running the
daemon.

\subsection{Screen Layout}

\subsection{Command Reference}

\subsection{Shooting}

\section{Ship Details}

\subsection{Cooling}

Firing weapons and running your engines generates heat which must be dissapated through
your ship's interlinked cooling system. As your systems heat up toward their maximum of
100% loaded, they try to dump their excess heat off into the opposite cooling system. This 
can be extremely useful when you are desperately trying to escape an enemy trap, and need
just that extra boost to get you into home space. Each weapon shot 'costs' given amount of
heat which is based on the power of the weapon. Warp drive on the other hand costs an amount
that increases as you increase your warp factor. With all of the above interactions taken into
account, it is usually possible for a normal ship to pull about warp five before starting
to accumulate heat.

In addition to these standard sources of heating if you are playing a configuration with
the teleport option, teleporting costs a large amount.

The cooling systems can only operate at peak efficiency with ship's shields down since the
entire purpose of the shield system is to repel energy. With shields up the cooling system
will still operate, but at a lower dissapation rate. Similarly, when the ship is cloaked the
cooling systems operate at a scaled back rate.

If the ships cooling system exceed 100% capacity, there is a growing probability during each
time slice that the system being cooled will shut down to protect itself. The capacity of the
cooling system must approach 0% for the system to be reactivated. When a system 'temps', a flag
is triggered on the status line, W for weapns, E for engines.

\subsection{Warp Factor}






\subsection{Status Area}

The status area contains your vessel's 'vital signs.'
If you are locked onto another ship, in addition to your own you
will get the status of the other vessel on the second line, minus
the \bold{Flags} field. There are eleven seperate sections to the
full status line :

\begin{itemize}
\item Flags
	This field contains information about ship activities.  There are twelve status characters
	that can be activated in this section (in order, left to right):

	\begin{itemize}

	\item \typewriter{S   }-- ship has shields up
	\item \typewriter{GYR }-- green, yellow, or red alert
	\item \typewriter{L   }-- ship has sensor or planet lock activated
	\item \typewriter{R   }-- ship is under repair}
	\item \typewriter{B   }-- ship is bombing a planet}
	\item \typewriter{O   }-- ship is orbiting a planet}
	\item \typewriter{C   }-- ship is cloaked}
	\item \typewriter{W   }-- ship's weapons are over operating temperature}
	\item \typewriter{E   }-- ship's engines are over operating temperature}
	\item \typewriter{u   }-- ship is beaming armies up from planet surface}
	\item \typewriter{d   }-- ship is beaming armies down to planet surface}
	\item \typewriter{P   }-- ship is allowing co-pilots}

	\end{itemize}

\item Warp -- shows the ship's current warp speed in x.x format

\item Damage -- shows the ship's current hull damage

\item Shield -- shows the number of damage points your shield
	generators can absorb

\item Torpedoes -- shows the number of torpedoes the ship has flying

\item Kills -- shows the number of kills the ship has prosecuted in x.xx format

\item Armies -- shows the number of armies currently onboard

\item Fuel -- shows the current amount of fuel onboard

\item Weapon Temp -- weapon temperature as a percentage of maxmimum

\item Engine Temp -- engine temperature as a percentage of maxmimum

\item Failed Systems
	This field contains info about which of the ship's systems are currently
	non-operational due to excessive damage.  If a letter is lit, the system
	is down (in order, left to right)

	\begin{itemize}

	\item \typewriter{C  }-- cloaking device
	\item \typewriter{L  }-- long range sensors
	\item \typewriter{P  }-- phaser banks
	\item \typewriter{S  }-- short range sensors 
	\item \typewriter{T  }-- torpedoes
	\item \typewriter{c  }-- cooling system
	\item \typewriter{l  }-- sensors (locking ablility)
	\item \typewriter{s  }-- shields
	\item \typewriter{t  }-- transporters

	\end{itemize}
\end{itemize}

\subsubsection{Armies}
\subsection{Ship Subsystems}
\subsubsection{Photon Torpedoes}
\subsubsection{Phasers}
\subsubsection{Shields}
\subsubsection{Engines}
\subsubsection{Cloak}
Note that engines and weapons cool 25\% slower when
cloak is up.

\subsubsection{Mines}
\subsubsection{Sensors}
\subsubsection{Transporter}

\section{Planets}

\subsection{Conquering Planets}

Planets are conquered by first bombing them to reduce the number of
armies to a manageable level and then beaming down your own armies.  A
ship must be in orbit to bomb a planet.  A side effect of this
requirement is that shields are dropped and planetary defensive fire
damages the hull directly.  Planetary fire causes damage proportional
to the number of defending armies.  The actual formula is
$(\mbox{armies}/ 10) + 2$ damage points twice per second.

\subsection{Fuel and Repair}
Fuel replenishes at twice the normal rate when orbiting a planet.  Certain
planets are marked as {\em fuel} sources;  orbiting these planets doubles
the rate again.

\section{Default Scenario}

%this is taken from Dan's informal description waay back in July 

Klingon : heavy phasers, low torps. normal engines, but they
can't slow down very fast, so turning is slower than normal. low fuel
gain-back, but they make up for that by toasting anything that gets
near with the phasers. best cloaking costwise ...

Romulan : nickname 'garbage scow', pretty accurate. slowest
turn rate in the game. heavy torps that will rail an opponent that
gets caught in a stream. best shields, and in terms of how much energy
they chew, best cloaking since they also have the best energy regen
rate. hull strength also highest.

Orion : tend to spit torps, which are generally weak but very
cheap and fast to fire. best acel/decel in the game, and turn on a
dime. hull pts and shield are the lowest, so they need those engines
of theirs often. regen is also low, so they need accuracy on those
torp shots. phasers are also weak, and don't have a very good range.
mostly useful for annoyance shots and close in fighting.

Federation : this is basically the standard xtrek ship of old
(we had to keep one of them around :-). All around average ship. High
cloak cost, but also have a decent regen rate. Best repair rate
(Scotty factor :-), and their weapons are both pretty potent. Can get
out-paced if they don't watch out, though - especially by Orions.

% Copyright 1994 NEC Corporation, Tokyo, Japan.
%
% Permission to use, copy, modify, distribute and sell this software
% and its documentation for any purpose is hereby granted without
% fee, provided that the above copyright notice appear in all copies
% and that both that copyright notice and this permission notice
% appear in supporting documentation, and that the name of NEC
% Corporation not be used in advertising or publicity pertaining to
% distribution of the software without specific, written prior
% permission.  NEC Corporation makes no representations about the
% suitability of this software for any purpose.  It is provided "as
% is" without express or implied warranty.
%
% NEC CORPORATION DISCLAIMS ALL WARRANTIES WITH REGARD TO THIS SOFTWARE,
% INCLUDING ALL IMPLIED WARRANTIES OF MERCHANTABILITY AND FITNESS, IN 
% NO EVENT SHALL NEC CORPORATION BE LIABLE FOR ANY SPECIAL, INDIRECT OR
% CONSEQUENTIAL DAMAGES OR ANY DAMAGES WHATSOEVER RESULTING FROM LOSS OF 
% USE, DATA OR PROFITS, WHETHER IN AN ACTION OF CONTRACT, NEGLIGENCE OR 
% OTHER TORTUOUS ACTION, ARISING OUT OF OR IN CONNECTION WITH THE USE OR 
% PERFORMANCE OF THIS SOFTWARE. 
%
% $Id: custom.tex,v 2.19 1994/06/01 06:32:29 kuma Exp $
%
\chapter{�������ޥ���}
\label{3.3�������ޥ���}

���ʴ����Ѵ��˴ؤ����Ѵ������Ѥ��뼭���Ϥ�Ȥ��ƥ�����������޻������Ѵ��ơ��֥�ˤ�����ޤ�\HIDX{�������ޥ���}{�������ޤ���}{H0.4.0.0.0}���뤳�Ȥ��Ǥ��ޤ���

\vspace{5mm}
%\begin{center}
\begin{tabular}{ll}
���ܸ����Ϥε�ǽ�䥭���������ꤹ�� &  \hpref{count=10,label=H0.4.1.0.0}�������ޥ����ե����� \\
�����޻������Ѵ��ε�§�����ꤹ��     &  \hpref{count=11,label=H0.4.4.0.0}�����޻������Ѵ������� \\
                                     &  ��������������������(�ߡߡ�.kpdef) \\
                                     &  \hpref{count=12,label=H0.4.4.0.0}�����޻������Ѵ��ơ��֥� \\
                                     &  ��������������������(�ߡߡ�.kp) \\
\end{tabular}
%\end{center}
\vspace{5mm}

�����ǤϤ��줾��Υ桼���˹�碌���������ޥ�����ˡ�ˤĤ����������ޤ���

% ------------------------------ 3.3.1
%\section{\HIDX{������ե�����(�������ޥ����ե�����)}
%{���褭���դ�����ʤ������ޤ����դ������}{H0.4.1.0.0}}
\section{\HIDX{������ե�����}{���褭���դ�����}{H0.4.1.0.0}
(\HIDX{�������ޥ����ե�����}{�������ޤ����դ�����}{H0.4.1.0.0})}
\label{3.3.1�����(�������ޥ���)�ե�����}

�������ޥ����Ͻ����(�������ޥ���)�ե�������Ѥ��ƹԤ��ޤ��������(��
�����ޥ���)�ե���������̤λ��꤬�ʤ�������ۡ���ǥ��쥯�ȥ��
���� \HIDX{.canna}{���ã��ΣΣ�}{H0.4.1.0.0} �Ȥ����ե����뤬�Ѥ����ޤ���

�Ķ��ѿ�������ˤ�껲�Ȥ�������(�������ޥ���)�ե�������ѹ����뤳�Ȥ�
�Ǥ��ޤ���
�����(�������ޥ���)�ե�����ϰʲ��ν�˥���������ޤ���

\begin{enumerate}

\item �Ķ��ѿ� \HIDX{CANNAFILE}{�ã��ΣΣ��ƣɣ̣�}{H0.4.1.0.0} ��
���ꤵ��Ƥ���Ȥ�

�Ķ��ѿ� CANNAFILE �ˤ�äƼ������ե����뤬¸�ߤ���С����Υե����뤬
�����(�������ޥ���)�ե�����Ȥ��ƻ��Ѥ���ޤ���

\item\label{.canna} \$HOME/.canna��¸�ߤ���Ȥ�

���Υե����뤬�����(�������ޥ���)�ե�����Ȥ��ƻ��Ѥ���ޤ���

\item\label{other} �嵭�Τ�����Ǥ�ʤ��Ȥ�

\refCANNALIBDIR /\HIDX{default.canna}{�ģţƣ��գ̣ԡ��ã��ΣΣ�}{H0.4.1.0.0} ��
�����(�������ޥ���)�ե�����Ȥ����Ѥ����ޤ���

\end{enumerate}

�ޤ���(\ref{.canna}), (\ref{other})�ξ���
���줾�켡�Υե�������ɤ߹��ޤ�ޤ���

\begin{itemize}
\item (\ref{.canna})��
\begin{nquote}{1em}
\begin{namelist}{������}
\item[{\dm \rm (2.1)}]�Ķ��ѿ� \HIDX{DISPLAY}{�ģɣӣУ̣���}{H0.4.1.0.0} ��
�ͤ����ꤵ��Ƥ��ꡢ�����ͤ� �ߡߡ�:0 �Ǥ���Ȥ��ˡ��ۡ���ǥ��쥯�ȥ�� 
~.canna-$\times\times\times$ �Ȥ���̾���Υե����뤬¸�ߤ����硢�����
���Υե�������ɤ߹��ޤ�ޤ���

\item[{\dm \rm (2.2)}]�Ķ��ѿ� \HIDX{TERM}{�ԣţң�}{H0.4.1.0.0} ��
�ͤ����ꤵ��Ƥ��ꡢ�����ͤ� $\times\times\times$ �Ǥ���Ȥ��ˡ�
�ۡ���ǥ��쥯�ȥ�� ~.canna-$\times\times\times$ �Ȥ���̾���Υե����뤬
¸�ߤ����硢����ˤ��Υե�������ɤ߹��ޤ�ޤ���
\end{namelist}
\end{nquote}

\item (\ref{other})��
\begin{nquote}{1em}
\begin{namelist}{������}
\item[{\dm \rm (3.1)}]�Ķ��ѿ� DISPLAY ���ͤ����ꤵ��Ƥ��ꡢ�����ͤ� 
$\times\times\times$:0 �Ǥ���Ȥ��� \refCANNALIBDIR �� 
$\times\times\times$.canna �Ȥ���̾���Υե����뤬¸�ߤ����硢
����ˤ��Υե�������ɤ߹��ޤ�ޤ���

\item[{\dm \rm (3.2)}]�Ķ��ѿ� TERM ���ͤ����ꤵ��Ƥ��ꡢ
�����ͤ� $\times\times\times$ �Ǥ���Ȥ��� \refCANNALIBDIR �� 
$\times\times\times$.canna �Ȥ���̾���Υե����뤬¸�ߤ����硢
����ˤ��Υե�������ɤ߹��ޤ�ޤ���
\end{namelist}
\end{nquote}
\end{itemize}

�ʾ��ɽ�ˤ���Ƚ����(�������ޥ���)�ե�����Υ�������ϰʲ��ΤȤ���ˤʤ�ޤ���

\vspace{5mm}

%{\tt
\begin{table}[hbtp]
\caption{�����(�������ޥ���)�ե�����Υ�������}
\label{�����(�������ޥ���)�ե�����Υ�������}
\begin{center}
\begin{tabular}{|l|l|}
\hline
   &  \multicolumn{1}{|c|}{�ե�����} \\
\hline
1  &   CANNAFILE �ǻ��ꤷ���ե����� \\
\hline
2  &   \$HOME/.canna \\
   &   \$HOME/.canna-X�Υǥ����ץ쥤̾ \\
   &   \$HOME/.canna-term̾ \\
\hline
3  &   \refCANNALIBDIR /default.canna \\
   &   \refCANNALIBDIR /X�Υǥ����ץ쥤̾.canna \\
   &   \refCANNALIBDIR /term̾.canna \\
\hline
\end{tabular}
\end{center}
\end{table}
%}

%\vspace{5mm}

�����(�������ޥ���)�ե������ \HIDX{Lisp}{�̣ɣӣ�}{H0.4.1.0.0} �����Ʊ�ͤ�
���󥿥å����ˤ�국�Ҥ��ޤ���
�ޤ�����;��(���ߥ�����)��������ޤǤϥ����ȤȤߤʤ���ޤ���

�ʲ���\HIDX{�������ޥ����ե��������}{�������ޤ����դ�����Τ줤}{H0.4.1.0.0}
�򼨤��ޤ���

\vspace{5mm}

{\tt 
\begin{center}
\begin{tabular}{|l|}
\hline
\verb+ ;;  +\\
\verb+ ;;�������ޥ�������  + \\
\verb+ ;; + \\
\verb+ (setq cursor-wrap                t)    ;�������뤬�۴Ĥ���褦�ˤ��롣 + \\
\verb+ (setq gakushu                    t)    ;�ؽ�����褦�ˤ��롣 + \\
\verb+ (setq kakutei-if-end-of-bunsetsu t)    ;ʸ��κǸ�DZ��˹Ԥ��ȳ��ꤹ�롣 + \\
\verb+ (setq break-into-roman           t)    ;BS�Ǥ��ä�������޻����᤹ + \\
\verb+ (setq grammatical-question       nil)  ;ñ����Ͽ�ΤȤ��ʻ�μ���򤷤ʤ��� + \\
\verb+ (setq kouho-count                t)    ;��������ΤȤ� 1/25 �ʤɤ�ɽ���򤹤롣 + \\
\verb+ (setq n-henkan-for-ichiran       3)    ;�Ѵ�����3��ǰ�����Ф��褦�ˤ��롣 + \\
\verb+   + \\
\verb+ ; ���Ѥ�������޻������Ѵ��ơ��֥������ + \\
\verb+  + \\
\verb+ (setq romkana-table   "default.kp") + \\
\verb+  + \\
\verb+ ; ���Ѥ��뤫�ʴ����Ѵ���������� + \\
\verb+  + \\
\verb+ (use-dictionary + \\
\verb+    "iroha" + \\
\verb+    "fuzokugo" + \\
\verb+    "hojoswd" + \\
\verb+    :bushu  "bushu"     ; �����Ѵ��Ѽ��� + \\
\verb+    )+ \\
\verb+ ; �����޻������Ѵ���¿�������� + \\
\verb+  + \\
\verb+ (defsymbol + \\
\verb+   ?["��" "�� " "[" "�� " "�� " + \\
\verb+   ?]"��" "��" "]" " ��" "��") + \\
\verb+  + \\
\verb+ (defsymbol + \\
\verb+   ?��"��" "��" "." + \\
\verb+   ?��"��" "��" ",") + \\
\verb+  + \\
\verb+ ;;��������� + \\
\verb+  + \\
\verb+ (global-set-key "\F1" 'kigou-mode) + \\
\verb+ (global-set-key "\F2" 'hex-mode) + \\
\verb+ (global-set-key "\F3" 'bushu-mode) + \\
\hline
\end{tabular}
\end{center}
}

\vspace{5mm}

\refCANNALIBDIR /sample �β��˥������ޥ����ե������
����ץ�(�ߡߡ�.canna)������ޤ��Τǻ��ͤˤ��Ƥ���������
default.canna ���ǥե���Ȥ�����򥫥����ޥ����ե������
ɽ������ΤˤʤäƤ��ޤ���

�ʲ��ǥ������ޥ����ε��ҤˤĤ����������ޤ���

% ------------------------------ 3.3.2
\section{���Ѥ���\HIDX{����λ���}{������Τ��Ƥ�}{H0.4.2.0.0}}
\label{3.3.2���Ѥ��뼭��λ���}

�����ܸ����ϥ����ƥ�Ǥ�ʣ���μ����Ʊ�������Ѥ��뤳�Ȥ��Ǥ��ޤ���
���Ѥ��뼭��λ�����ˡ�ϡ����μ����ɤΤ褦�����Ѥ��뤫�ˤ�ꡢ
ɽ\ref{������ꥭ�����}�˼��� 4 ���ढ��ޤ���

%\vspace{5mm}

%{\tt
\begin{table}[hbtp]
\begin{center}
\caption{������ꥭ�����}\label{������ꥭ�����}
\begin{tabular}{|l|l|p{7cm}|} \hline
\multicolumn{1}{|c|}{�������} & \multicolumn{1}{|c|}{����μ���} & \multicolumn{1}{|c|}{����} \\ \hline
\multicolumn{1}{|c|}{��} & �����ƥ༭��   & �����ƥ༭��Ȥ����Ѥ��뼭�����ꤷ�ޤ������μ���ˤ�ñ�����Ͽ�ϹԤ��ޤ��� \\ \hline
:bushu                   & �����Ѵ��Ѽ��� & �����Ѵ����Ѥ��뼭�����ꤷ�ޤ��� \\ \hline
:grammar                 & ʸˡ�����Ѽ��� & �ʻ��������ʻ�֤���³�������ļ������ꤷ�ޤ��� \\ \hline
:user                    & ñ����Ͽ�Ѽ��� & ñ����Ͽ�Ѥμ������ꤷ�ޤ������μ���ˤ�ñ�����Ͽ���뤳�Ȥ��Ǥ��ޤ������μ���η����ϥƥ����ȷ����Ǥʤ���Фʤ�ޤ���\raisebox{0.6ex}{\footnotesize \dg ��} \\ \hline
\end{tabular}
\end{center}
\end{table}
%}

%\vspace{5mm}

\begin{nquote}{1em}
\begin{namelist}{��}
\item[����]�ƥ����ȷ�������
�Х��ʥ��������ˤĤ��Ƥ� {\dg\bf \ref{���⡦���ʴ����Ѵ�����} 
\hpref{count=8,label=H0.3.1.2.0}���ʴ����Ѵ�����} ����� {\dg\bf
\ref{3.4.2����} \hpref{count=2,label=H0.5.2.0.0}����} �򻲾Ȥ��Ƥ���������
\end{namelist}
\end{nquote}

���Ѥ��뼭��λ���ϥ������ޥ����ե��������Ǽ��Τ褦�˹Ԥ��ޤ���

\HIDXAS{use-dictionary}{�գӣšݣģɣãԣɣϣΣ��ң�}{H0.4.2.0.0}
\begin{nquote}{3em}
\begin{verbatim}
(use-dictionary   "iroha"
                  "fuzokugo"
                  "hojomwd"
                  "hojoswd"
                  "yuubin"
 :bushu           "bushu"
 :grammar         "grammar"
 :user            "mine" )
\end{verbatim}
\end{nquote}

\begin{nquote}{3em}
\begin{namelist}{����}
\item[����] use-dictionary �ϰ��٤ε��ҤǤ��������
�������ꤷ�Ƥ�$^{(1)}$��ʣ����ε��Ҥ�ʬ���ơ�
�������ꤷ�Ƥ�$^{(2)}$Ʊ�����̤������ޤ���
���ʤ�����ʲ��� (1) �� (2) ��Ʊ����̣������ޤ���
\begin{nquote}{3em}
\begin{enumerate}
\item (use-dictionary "A"  "B"  "C")
\item (use-dictionary "A") \\
(use-dictionary "B") \\ (use-dictionary "C")
\end{enumerate}
\end{nquote}
\end{namelist}

\end{nquote}

����Ǥ��뼭��Ȥ��Ƥ�ɽ\ref{����Ǥ��뼭��}�˼�����Τ�����ޤ���

%\vspace{5mm}

%{\tt
\begin{table}[hbtp]
\begin{center}
\caption{����Ǥ��뼭��}\label{����Ǥ��뼭��}
\begin{tabular}{|l|l|}
\hline
\multicolumn{1}{|c|}{�� �� ̾}    & \multicolumn{1}{|c|}{�⡡����������} \\
\hline
 %%%%% %%%%%% %%%%% %%%%%% %%%%% %%%%%%
 %iroha     & �𴴼���(8����) \\
iroha     & \HIDX{�𴴼���}{�����󤷤���}{H0.4.2.0.0} \\
\hline
fuzokugo  & \HIDX{��°�켭��}{�դ�����������}{H0.4.2.0.0} \\
\hline
bushu     & \HIDX{���󼭽�}{�դ��椷����}{H0.4.2.0.0} \\
\hline
yuubin    & \HIDX{͹���ֹ漭��}{�椦�Ҥ�Ϥ󤳤�������}{H0.4.2.0.0} \\
\hline
hojomwd   & ����ɽ���ʤɤ����Ū�ʼ�Ω�� \\
\hline
hojoswd   & ����ɽ���ʤɤ����Ū����°�� \\
\hline
 %%%%% %%%%%% %%%%% %%%%%% %%%%% %%%%%% 
 %necgaiji  & �����ŵ��γ�����ޤ�� \\
 %\hline
 %%%%% %%%%%% %%%%% %%%%%% %%%%% %%%%%%
\end{tabular}
\end{center}
\end{table}
%}

%\vspace{5mm}

�����μ���Τ�����iroha ������Ū�ʼ���ˤʤ�ޤ���
�ޤ���iroha ����Ѥ���Ȥ��� fuzokugo ��ɬ�����ꤷ�ʤ���Фʤ�ޤ���
�������äƺ���»��Ѥ��뼭��� iroha ����� fuzokugo �Ǥ��ꡢ
���Τ褦�˻��ꤹ�뤳�Ȥˤʤ�ޤ���

\begin{nquote}{3em}
\begin{verbatim}
(use-dictionary  "iroha"  "fuzokugo" )
\end{verbatim}
\end{nquote}

͹���ֹ漭���͹���ֹ椫����̾�ؤ��Ѵ���Ԥ�����μ���Ǥ���
ɬ�פ˱����Ƥ����Ѥ���������

hojomwd,hojoswd�����Ū���ɲä��줿��°��μ���Ǥ�������ɽ���ʤɤϤ��μ������Ѥ��뤳�Ȥˤ���Ѵ����פ��ʤ�ޤ���ɬ�פ˱����Ƥ����Ѥ���������

�嵭�� 3 �Ĥμ���ϰʲ��Τ褦�˻��ꤹ�뤳�Ȥˤ�����ѤǤ��ޤ���

\begin{nquote}{3em}
\begin{verbatim}
(use-dictionary  "yuubin"  "hojomwd"  "hojoswd" )
\end{verbatim}
\end{nquote}

bushu �������Ѵ���Ԥ��Ȥ��˻Ȥ��뼭��Ǥ���
�����Ѵ������Ѥ�����Ϥ��μ����Ȥ����Ȥ�ʲ��Τ褦�ˤ��ƻ��ꤷ�Ƥ���������
����λ���Τ���Υ������ \HIDX{:bushu}{���£գӣȣ�}{H0.4.2.0.0} ��
ɬ�פǤ��Τǡ������դ���������

\begin{nquote}{3em}
\begin{verbatim}
(use-dictionary  :bushu  "bushu" )
\end{verbatim}
\end{nquote}

�ʻ�����������ʻ�֤���³�������ļ������ꤹ�뤿��ˤϡ�
������� \HIDX{:grammar}{���ǣң��ͣͣ���}{H0.4.2.0.0} ���Ѥ��ޤ���
:grammar �ǻ��ꤵ�줿���񤬤ʤ���硢
����� :grammar �ǻ��ꤵ�줿�������°����󤬵��Ҥ���Ƥ��ʤ����ϡ�
fuzokugo.d �˵��Ҥ���Ƥ���ʸˡ�������Ѥ��ޤ���
�ޤ���:grammar �� 2 �İʾ���ꤵ�줿���ϡ���˻��ꤵ�줿�������Ѥ��ޤ���
:grammar �λ���ϰʲ��Τ褦�˹Ԥ��ޤ���

\begin{nquote}{3em}
\begin{verbatim}
(use-dictionary :grammar "myGrammarDictionary" )
\end{verbatim}
\end{nquote}

�桼�����Ȥ��ɲ���Ͽ����ñ���Ǽ��Ƥ�������μ������ꤹ�뤿��ˤϡ�
������� \HIDX{:user}{���գӣţ�}{H0.4.2.0.0} ���Ѥ��ޤ���
ñ����Ͽ�κݡ���Ͽ���줿ñ���ɤμ����Ǽ��뤫��Ҥͤ��ޤ�����
���ΤȤ�����Ǥ��뼭��� :user �ǻ��ꤷ������˸¤��ޤ���

:user �ǻ��ꤹ�뼭���ñ����Ͽ��Ԥ��ط��塢ñ����Ͽ�Ѽ���Ȥ��ƺ������줿
�ƥ����ȷ����μ���Ǥʤ���Фʤ�ޤ���\raisebox{0.6ex}{\footnotesize \dg ��}�� 
:user �λ���ϰʲ��Τ褦�˹Ԥ��ޤ���

\begin{nquote}{3em}
\begin{verbatim}
(use-dictionary :user "myTextDictionary" )
\end{verbatim}
\end{nquote}

\begin{nquote}{1em}
\begin{namelist}{��}
\item[����]�ƥ����ȷ������񡢥Х��ʥ��������ˤĤ��Ƥ� 
{\dg\bf \ref{���⡦���ʴ����Ѵ�����} \hpref{count=8,label=H0.3.1.2.0}���ʴ����Ѵ�����} ����� 
{\dg\bf \ref{3.4.2����}\hpref{count=2,label=H0.5.2.0.0}����} �򻲾Ȥ��Ƥ���������
\end{namelist}
\end{nquote}

ñ����Ͽ�Ѽ���� mkdic ���ޥ�ɤǺ����Ǥ��ޤ���
�ǽ�϶��Υե��������ꤷ�Ƥ�����
�缡������Ͽ�ˤ��ñ����ɲä��Ƥ����Τ��褤�Ǥ��礦��

�����ѤȤ��ƥ����ƥ�ե�����(\refCANNALIBDIR /dic/user/user)��
�ޤޤ�Ƥ��뼭���ɽ\ref{user�ǥ��쥯�ȥ�ˤ���ե�����}�˼����ޤ���
ñ����Ͽ�Ѽ�����������Ȥ��λ��ͤˤ��Ƥ���������

%\vspace{5mm}

%{\tt
\begin{table}[hbtp]
\begin{center}
\caption{\refCANNALIBDIR /dic/user/user �ǥ��쥯�ȥ�ˤ���ե�����}\label{user�ǥ��쥯�ȥ�ˤ���ե�����}
\begin{tabular}{|l|l|}
\hline
\multicolumn{1}{|c|}{�� �� ̾} & \multicolumn{1}{|c|}{�⡡������} \\
\hline
katakana  & �������ʸ� \\
\hline
software  & ���եȥ������졡������ \\
\hline
chimei    & ��̾ \\
\hline
\end{tabular}
\end{center}
\end{table}
%}

\begin{nquote}{1em}
\begin{namelist}{��}
\item[����]�ǥե���Ƚ����(�������ޥ���)�ե�����(\refCANNALIBDIR/default.canna)�Ǥϡ�ñ����Ͽ�Ѽ���Ȥ��Ƥ��餫���� "user" �����ꤵ��Ƥ��ޤ���
\end{namelist}
\end{nquote}

% ------------------------------ 3.3.3
\section{\HIDX{�����޻������Ѵ��ơ��֥������}{�����ޤ����ʤؤ󤫤�ơ�
�դ�Τ��ĤƤ�}{H0.4.3.0.0}}
\label{3.3.3�����޻������Ѵ��ơ��֥������}

���Ѥ����������޻������Ѵ��ơ��֥�򵭽Ҥ��ޤ���

\HIDXAS{romkana-table}{�ңϣͣˣ��Σ��ݣԣ��£̣�}{H0.4.3.0.0}
\begin{nquote}{3em}
\begin{verbatim}
(setq romkana-table  "romaji.kp")
\end{verbatim}
\end{nquote}

romkana-table �˻��ꤹ���Τϥե�����̾�Ǥ���
���ꤵ�줿�ե�����̾�򥫥��ȥǥ��쥯�ȥꡢ
�桼���Υۡ���ǥ��쥯�ȥꡢ\refCANNALIBDIR /dic �ν�˥��������� 
�ǽ�˸��Ĥ��ä��ե����뤬�����޻������Ѵ��ơ��֥�Ȥ����Ѥ����ޤ���

% ------------------------------ 3.3.4
\section{\HIDX{�����޻������Ѵ��ơ��֥�Υ������ޥ���}{�����ޤ����ʤ�
�󤫤�ơ��դ�Τ������ޤ���}{H0.4.4.0.0}}
\label{3.3.4�����޻������Ѵ��ơ��֥�Υ������ޥ���}

�����޻������Ѵ��˴ؤ��ơ����Ȥ��С֤���ˤ��ϡפ�
���Ϥ���Τˡ���konnichiha�פ����Ϥ���Τ˴���Ƥ�����ȡ���konnnichiha��
�����Ϥ���(��n�פβ�����㤤�ޤ�)�Τ˴���Ƥ����礬����ޤ���

���Τ褦�˥����޻������Ѵ�������ȤäƤߤƤ�桼���ι��ߤ�
������������ޤ���

�����޻������Ѵ��˴ؤ��ƤϽ����(�������ޥ���)�ե�����ˤ��
���ߤΥơ��֥����ꤹ�뤳�Ȥ��Ǥ��ޤ���
�ޤ���ɸ��Ū���Ѱդ��Ƥ���ơ��֥�˹��ߤΤ�Τ��ʤ����ϡ�
��ʬ�ǥ����޻������Ѵ��ơ��֥��������뤳�Ȥ�Ǥ��ޤ���

% ------------------------------ 3.3.4.1
\subsection{ɸ��Ū���󶡤�������޻������Ѵ��ơ��֥�}
\label{3.3.4.1ɸ��Ū���󶡤�������޻������Ѵ��ơ��֥�}

ɸ��Ū���󶡤��������޻������Ѵ��ơ��֥�Ȥ��Ƥ�
�ʲ��Τ�Τ�����ޤ��� �����Υơ��֥�� \refCANNALIBDIR /dic �β��˥ե�
����η���¸�ߤ��ޤ���

%\vspace{5mm}

%{\tt
\begin{table}[hbtp]
\begin{center}
\caption{�󶡤�������޻�����ơ��֥�}\label{�󶡤�������޻�����ơ��֥�}
\begin{tabular}{|l|p{10cm}|}
\hline
\multicolumn{1}{|c|}{����̾} &  \multicolumn{1}{|c|}{������������������} \\
\hline
default.kp  & �ǥե���ȤΥ����޻������Ѵ��ơ��֥� \\
\hline
 %%%%% %%%%% %%%%% %%%%% %%%%% %%%%%
 %jdaemon.kp  & EWS-UX/V R6.�߰��������ܸ����ϥ����ƥ�ȸߴ��Υ����޻������Ѵ��ơ��֥� \\
 %\hline
 %%%%% %%%%% %%%%% %%%%% %%%%% %%%%%
just.kp     & ����Ϻ\footnotemark ��Ʊ��Υ����޻������Ѵ���§�����
�����޻������Ѵ��ơ��֥� \\
\hline
kana.kp     & ����ե��٥åȥ����ܡ��ɤǵ���Ū��
�������Ϥ�Ԥ�����Υơ��֥롣���Υơ��֥���Ѥ��뤳�Ȥˤ�ꡢ
���ܸ�⡼�ɤȥ���ե��٥åȥ⡼�ɤ��ڤ��ؤ������ǡ�
�������Ϥȥ���ե��٥å����Ϥ��ڤ��ؤ���Ԥ��ޤ��� \\
\hline
newjis.kp   & ����ե��٥åȥ����ܡ��ɤǵ���Ū�˿�JIS�����ܡ��ɤΥ�������򥷥ߥ�졼�Ȥ��뤿��Υơ��֥� \\
\hline
kaisoku.kp  & ��®�����޻�����򥷥ߥ�졼�Ȥ��뤿��Υơ��֥� \\
\hline
\end{tabular}
\end{center}
\end{table}
%}

\footnotetext{����Ϻ�ϥ��㥹�ȥ����ƥ�(��)�ξ�ɸ�Ǥ�}


%\vspace{5mm}

�ǥե���ȤǤ�default.kp�����Ф�ޤ��������Τۤ��Υ����޻������Ѵ��ơ��֥��
���Ѥ�����Ͻ����(�������ޥ���)�ե�����˼��Τ褦�ʹԤ�ä��ޤ���

\begin{nquote}{3em}
\begin{verbatim}
;kana.kp�����Ѥ�����
(setq romkana-table "kana.kp")
\end{verbatim}
\end{nquote}

���ꤷ��̾���ϥե�����̾�Ȥߤʤ��졢���ν�˥���������ޤ���

\begin{nquote}{3em}
\begin{namelist}{����}
\item[(1) �����ȥǥ��쥯�ȥ�] ��\\
���ꤵ�줿�ե�����򥫥��ȥǥ��쥯�ȥ�ǥ��������ޤ���
\item[(2) �ۡ���ǥ��쥯�ȥ�]�� \\
���ꤵ�줿�ե������ۡ���ǥ��쥯�ȥ�ǥ��������ޤ���
\item[(3) ����ǥ��쥯�ȥ�]�� \\
���ꤵ�줿�ե������\refCANNALIBDIR /dic�ǥ��������ޤ���
\end{namelist}
\end{nquote}

������Υǥ��쥯�ȥ�ˤ�����޻������Ѵ��ơ��֥뤬¸�ߤ��ʤ�����
�����޻������Ѵ����Ϥ��Ԥ��ޤ���ΤǤ����դ���������

% ------------------------------ 3.3.4.2
\subsection{\HIDX{�����޻������Ѵ��ơ��֥�κ���}
{�����ޤ����ʤؤ󤫤�ơ��դ�Τ�������}{H0.4.4.2.0}��ˡ}
\label{3.3.4.2��ʬ���ȤΥ����޻������Ѵ��ơ��֥�κ����λ���}

�����ƥ���󶡤���Ƥ�������޻������Ѵ��ơ��֥�Ǥ���­�Ǥ��ʤ����ϡ�
�����޻������Ѵ��ơ��֥�򥫥����ޥ������뤳�Ȥ��Ǥ��ޤ���

�������ޥ���������ϡ������޻������Ѵ��ơ��֥��
�������ե����� \refCANNALIBDIR /\\
sample/src/dafault.kpdef�򥳥ԡ����ƽ񤭴�����Τ���ñ�Ǥ���

�����޻������Ѵ��ơ��֥��1�Ԥ˥����޻��Ȥ�����б�����ؤ��ʡ٤Ȥ����Ҥ���Ƥ���ơ��֥�Ǥ������Τ褦�����ƤˤʤäƤ��ޤ���

\vspace{5mm}

{\tt
\begin{center}
\begin{tabular}{|lll|}
\hline
a   & ��       &     \\
i   & ��       &     \\
u   & ��       &     \\
e   & ��       &     \\
o   & ��       &     \\
ka  & ��       &     \\
ki  & ��       &     \\
    & �������� &     \\
n   & ��       &     \\
n'  & ��       &     \\
mn  & ��       &     \\
nn  & ��       &     \\
    & �������� &     \\
tch & ��       & ch������\\
kk  & ��       & k     \\
ss  & ��       & s     \\
tt  & ��       & t     \\
hh  & ��       & h     \\
\hline
\end{tabular}
\end{center}
}

\vspace{5mm}


��Ʊ���Ҳ����Ťʤä�����¥����ȯ������פȤ��ä���§��
�����޻������Ѵ��ơ��֥�˵��Ҥ��ޤ���
���Ȥ��С��ʲ��Τ褦�˵��Ҥ��ޤ���

\vspace{5mm}

{\tt
\begin{center}
\begin{tabular}{|p{6cm}|}
\hline
\multicolumn{1}{|c|}{ kk�������k������} \\
\hline
\end{tabular}
\end{center}
}

\vspace{5mm}

����ϡ�kk�����Ϥ��褿�Ȥ��ϡ֤áפ�ȯ��������k ��
�������ϤȤĤʤ��뤿��˻Ĥ��Ƥ����Ȥ�����̣�Ǥ���

���ε�§���Ѥ��ơ��֤ޤä��פ����Ϥ���Ȥ��ˡ�matchi�פΤ褦��
���Ϥ��뤿��Υ롼���ʲ��Τ褦�˵��Ҥ��뤳�Ȥ��Ǥ��ޤ���

\vspace{5mm}

{\tt
\begin{center}
\begin{tabular}{|p{6cm}|}
\hline
\multicolumn{1}{|c|}{tch�������ch������} \\
\hline
\end{tabular}
\end{center}
}

\vspace{5mm}

�ü��ʸ������ꤹ��Ȥ��Τ���˥Хå�����å���($\backslash$)��
����������ʸ���Ȥ��ƻ��Ѥ��뤳�Ȥ��Ǥ��ޤ���
�ü��ʸ���Ȥ��Ƥϡ����ڡ���ʸ����
���㡼��(\#),�Хå�����å���($\backslash$)������ޤ���

���Ȥ��С��Хå�����å���($\backslash$)�����Ϥ����Ȥ��� �� �����Ϥ����
�褦�ˤ�����ϡ��ʲ��Τ褦�˵��Ҥ��ޤ���

\vspace{5mm}

{\tt
\begin{center}
\begin{tabular}{|p{6cm}|}
\hline
\multicolumn{1}{|c|}{$\backslash \backslash $ �����} \\
\hline
\end{tabular}
\end{center}
}

\subsubsection{�ơ��֥�˵��Ҥ���Ƥ��뵬§���Ѥ������}
\HIDXAS{�Ѵ���§}{�ؤ󤫤󤭤���}{H0.4.4.2.1}

�����޻������Ѵ��ơ��֥���˥����޻������Ѵ����Ԥ��ޤ�����
�����޻������Ѵ��ơ��֥�ǵ��Ҥ��줿�Ѵ���§�ϡ���Ĺ����ˡ�θ�§�ǻ��Ѥ���ޤ���
���Ȥ��С��֤٤��ʡפ����Ϥ���Ȥ��ϼ��Τ褦�˥ơ��֥뤬���Ȥ���ޤ���

%\vspace{5mm}

{\tt
\begin{center}
\begin{tabular}{|l|l|p{8cm}|}
\hline
\multicolumn{1}{|c|}{����} & \multicolumn{1}{|c|}{ɽ����} & \multicolumn{1}{|c|}{����������������}\\
\hline
ben   &  ��n       & ��n�פˤĤ��ƤϤ�ä�Ĺ����§�Ǥ����na�פΤ褦�ʤ�Τ�¸�ߤ���Τǥ����޻��Τޤޥ���������ޤ��� \\
\hline
r     &  �٤�r     & ��nr�פǻϤޤ뵬§���ʤ��Τǡ�n�ע��֤�פε�§��Ŭ�Ѥ���ޤ� \\
\hline
i    &  �٤��    &  \\
\hline
n    &  �٤��n    &  \\
\hline
a     &  �٤���  & ��na�פǻϤޤ뵬§����na�ע��֤ʡפ����Ǥ���ΤǤ��ε�§��Ŭ�Ѥ���ޤ��� \\
\hline
\end{tabular}
\end{center}
}



\subsubsection{�����޻������Ѵ��ơ��֥��\HIDX{����ѥ���}{����Ϥ���}
{H0.4.4.2.2}}

�����޻������Ѵ��ơ��֥�ϥ����ƥब�ɤ߰פ��Х��ʥ������
�Ѵ����ʤ���Фʤ�ޤ���
���ʤ�����嵭�Τ褦�ʥƥ����ȷ����Υե������
�Х��ʥ�����˥���ѥ��뤹��ɬ�פ�����ޤ���

���Τ褦�ˤ��ơ������޻������Ѵ��ơ��֥��ƥ����ȷ�������
�Х��ʥ�����˥���ѥ��뤷�Ƥ���������

\begin{nquote}{3em}
  \%��mkromdic���ե�����̾ \RETURN
\end{nquote}

mkromdic ��Ԥ��ȡ�$\ast$.kp �ե����뤬�Ǥ��ޤ���

�����ƥ���󶡤��������޻������Ѵ��ơ��֥�Υ������ե����뤬
\refCANNALIBDIR /sample/src ���
$\ast$.kpdef �Ȥ���¸�ߤ��ޤ��Τǻ��Ȥ��Ƥ���������

\begin{nquote}{1em}
\begin{namelist}{��}
\item[����]�Х��ʥ�����Υ����޻������Ѵ��ơ��֥�� dpromdic ���ޥ�ɤˤ��
�ƥ����ȷ������᤹���Ȥ���ǽ�Ǥ����ܺ٤ϡ�{\dg\bf \ref{4���ʴ�����
���桼�ƥ���ƥ�} \hpref{count=13,label=H0.6.0.0.0}���ʴ����Ѵ��桼�ƥ���ƥ�} �򻲾Ȥ��Ƥ���������
\end{namelist}
\end{nquote}

% ------------------------------ 3.3.5
\section{\HIDX{�������Υ������ޥ���}{�����������Τ������ޤ���}{H0.4.5.0.0}}
\label{3.3.5�������Υ������ޥ���}

���Ȥ��С����ܸ�⡼�ɤȥ���ե��٥åȥ⡼�ɤ��ڤ��ؤ��륭�����ǥե����
�Ǥ� \XFER �ޤ��� \CTRL + \fbox{o} �Ǥ�����
���ѼԤϤ��Υ����򹥤ߤ˱������ѹ����뤳�Ȥ��Ǥ��ޤ���
�����(�������ޥ���)�ե�������Ѥ��ơ����ʴ����Ѵ����Ѥ����륭���γ�����Ƥ�
�ǥե���ȤȤϰۤʤ륭���˳������ľ�����Ȥ��Ǥ��ޤ���

\subsection{�����γ�����Ƥ�ͭ���ϰ�}

���ʴ����Ѵ��Ǥ�¿���Υ⡼�ɤ����ܤ�����ޤ��������γ�����Ƥϳƥ⡼��
���Ȥ˸��̤��ѹ����뤳�Ȥ��Ǥ��ޤ����ޤ������٤ƤΥ⡼�ɤΥ����γ����
�Ƥ��礷���ѹ����뤳�Ȥ�Ǥ��ޤ���

\subsection{�ѹ��Ǥ��뵡ǽ}

�����γ�����Ƥ��ѹ��ϥ���̾���Ф���\HIDX{��ǽ̾}{���Τ��ᤤ}{H0.4.5.2.0}��
���Ҥ��뤳�Ȥˤ��Ԥ��ޤ���
��ʵ�ǽ̾�Ȥ��Ƥϼ�ɽ�Τ�Τ�����ޤ���
���Τۤ��ε�ǽ̾�˴ؤ��Ƥ�
{\dg ��Ͽ \ref{C�������ޥ������Ѥ��뵡ǽ̾����ɽ} \hpref{count=16,label=H0.C.0.0.0}�������ޥ������Ѥ��뵡ǽ̾����ɽ} �򻲾Ȥ��Ƥ���������

%������ʵ�ǽ̾

%{\tt
{\small
\begin{table}[hbtp]
\begin{center}
\caption{��ʵ�ǽ̾}\label{��ʵ�ǽ̾}
\begin{tabular}{|l|l|c|c|}
\hline
\multicolumn{1}{|c|}{̾������} & \multicolumn{1}{|c|}{������������ǽ} &\multicolumn{1}{|c|}{�ǥե����} & \multicolumn{1}{|c|}{���� ��} \\
\hline
\HIDX{alpha-mode}{���̣Уȣ��ݣͣϣģ�}{H0.4.5.2.0} 
		& ����ե��٥åȥ⡼�ɤ˰ܹԤ���& Xfer,C-o & \\
\hline
\HIDX{quoted-insert}{�ѣգϣԣţġݣɣΣӣţң�}{H0.4.5.2.0} 
		& ���ΰ�ʸ����̵�������Ϥ���  & C-q & ��\\
\hline
\HIDX{kakutei}{�ˣ��ˣգԣţ�}{H0.4.5.2.0} 
		& ���ꤹ��                      & Return, Nfer & ��\\
\hline
\HIDX{extend}{�ţأԣţΣ�}{H0.4.5.2.0} 
		& �ΰ迭�Ф�                    & S-��, C-o & ��\\
\hline
\HIDX{shrink}{�ӣȣңɣΣ�}{H0.4.5.2.0} 
		& �ΰ�̤�                      & S-��, C-i & ��\\
\hline
\HIDX{touroku}{�ԣϣգңϣˣ�}{H0.4.5.2.0} 
		& ñ����Ͽ                      & Help & ��\\
\hline
\HIDX{forward}{�ƣϣңף��ң�}{H0.4.5.2.0} 
		& ������                        & ��, C-f & ��\\
\hline
\HIDX{backward}{�£��ãˣף��ң�}{H0.4.5.2.0} 
		& ������                        & ��, C-b & ��\\
\hline
\HIDX{previous}{�Уңţ֣ɣϣգ�}{H0.4.5.2.0} 
		& ������                        & ��, C-p & ��\\
\hline
\HIDX{next}{�Σţأ�}{H0.4.5.2.0} 
		& ������                        & ��, C-n & ��\\
\hline
\HIDX{beginning-of-line}{�£ţǣɣΣΣɣΣǡݣϣơݣ̣ɣΣ�}{H0.4.5.2.0} 
		& ��Ƭ����                      & C-a & ��\\
\hline
\HIDX{end-of-line}{�ţΣġݣϣơݣ̣ɣΣ�}{H0.4.5.2.0} 
		& ��������                      & C-e & ��\\
\hline
\HIDX{delete-next}{�ģţ̣ţԣšݣΣţأ�}{H0.4.5.2.0} 
		& ����ʸ���õ�                  & C-d & ��\\
\hline
\HIDX{delete-previous}{�ģţ̣ţԣšݣУңţ֣ɣϣգ�}{H0.4.5.2.0} 
		& ����ʸ���õ�                  & Backspace & ��\\
\hline
\HIDX{kill-to-end-of-line}{�ˣɣ̡̣ݣԣϡݣţΣġݣϣơݣ̣ɣΣ�}{H0.4.5.2.0} 
		& �����ޤǾõ�                  & C-k & ��\\
\hline
\HIDX{henkan}{�ȣţΣˣ���}{H0.4.5.2.0} 
		& �Ѵ�                          & Space, Xfer & ��\\
\hline
\HIDX{quit}{�ѣգɣ�}{H0.4.5.2.0}
		& �����                      & C-g & ��\\
\hline
\HIDX{self-insert}{�ӣţ̣ơݣɣΣӣţң�}{H0.4.5.2.0} 
		& ʸ������                      & a, b, c �� & ��\\
\hline
\end{tabular}
\end{center}
\end{table}
}	%% small no tame
%}

\vspace{5mm}

\subsection{��ǽ�ؤ�\HIDX{�����γ������}{�����Τ�ꤢ��}{H0.4.5.3.0}}
\label{4.5.2��ǽ�ؤΥ����γ������}

���뵡ǽ���Ф��ƥ����������Ƥ�Ȥ��ϡ�global-set-key �ޤ��� set-key ��
�Ѥ��ơ��ʲ��Τ褦�˵��Ҥ��뤳�Ȥˤ�������Ƥޤ���
���뵡ǽ���Ф���ʣ���Υ����������Ƥ���ϡ�
{\dg\bf \ref{4.5.5ʣ�������ǤΥ����γ������} \hpref{count=13,label=H0.4.5.4.0}ʣ�������ǤΥ����γ������} ��
���Ȥ��Ƥ���������

\HIDXAS{global-set-key}{�ǣ̣ϣ£��̡ݣӣţԡݣˣţ�}{H0.4.5.3.0}
\begin{nquote}{3em}
\begin{verbatim}
(global-set-key  "����"   '��ǽ̾)
\end{verbatim}
\end{nquote}

\HIDXAS{set-key}{�ӣţԡݣˣţ�}{H0.4.5.3.0}
\begin{nquote}{3em}
\begin{verbatim}
(set-key  '�⡼��̾  "����"  '��ǽ̾)
\end{verbatim}
\end{nquote}

global-set-key �Ǥϡ����ܸ����ϻ��Τ��٤ƤΥ⡼�ɤ��Ф���
�����γ�����Ƥ��Ԥ��ޤ���
set-key �Ǥϡ�����Υ⡼�ɤ��Ф��ƤΤߡ������γ�����Ƥ��Ԥ��ޤ���

set-key ����ꤹ�뤳�Ȥˤ�ꡢ����Υ⡼�ɤǤΥ����γ�����Ƥ�
���Υ⡼�ɤΤȤ�����ñ�Ȥ��ѹ����뤳�Ȥ��Ǥ��ޤ���


\HIDX{����}{����}{H0.4.5.3.0}�� \verb!"a"! �� \verb!"b"! �Τ褦�˻��ꤷ�ޤ���
����ȥ����륭���򲡤��ʤ��鲿�餫�Υ����򲡤��Ȥ����Τ�
�����λ�������� \verb!"\C-"! ���դ��뤳�Ȥ�ɽ�����ޤ���
�ե��󥯥���󥭡��ʤɤ�\HIDX{�ü쥭��}{�Ȥ����椭��}{H0.4.5.3.0}�ˤĤ��Ƥ�ɽ
\ref{�ü쥭������}�˼���̾���ǻ��ꤹ�뤳�Ȥ��Ǥ��ޤ���

������̾���ϡ���ʸ������ʸ������̤��ޤ��������դ���������

%\vspace{5mm}

%{\tt
{\small
\begin{table}[hbtp]
\begin{center}
\caption{�ü쥭���μ���}\label{�ü쥭������}
\begin{tabular}{|l|l|}
\hline
\multicolumn{1}{|c|}{��������} & \multicolumn{1}{|c|}{̾��������} \\
\hline
f��1(�ʲ�2,3��)   & \tt "$\backslash$F1",~"$\backslash$F2",~\ldots \\
 %%%%% %%%%% %%%%% %%%%% %%%%% %%%%%
 %Ʃ��(������1,2,��)& \tt "$\backslash$Pf1",~"$\backslash$Pf2",~\ldots \\
 %%%%% %%%%% %%%%% %%%%% %%%%% %%%%%
ESC               & \tt "$\backslash$Escape" \\
TAB               & \tt "$\backslash$Tab" \\
NFER              & \tt "$\backslash$Nfer" \\
XFER              & \tt "$\backslash$Xfer" \\
BS                & \tt "$\backslash$Backspace" \\
INS               & \tt "$\backslash$Insert" \\
DEL               & \tt "$\backslash$Delete" \\
ROLLUP            & \tt "$\backslash$Rollup" \\
ROLLDOWN          & \tt "$\backslash$Rolldown" \\
��                & \tt "$\backslash$Up" \\
��                & \tt "$\backslash$Left" \\
��                & \tt "$\backslash$Right" \\
��                & \tt "$\backslash$Down" \\
HOME              & \tt "$\backslash$Home" \\
CLR               & \tt "$\backslash$Clear" \\
HELP              & \tt "$\backslash$Help" \\
ENTER             & \tt "$\backslash$Enter" \\
(SPACE)           & \tt "$\backslash$Space" \\
(RETURN)          & \tt "$\backslash$Return" \\
\hline
\end{tabular}
\end{center}
\end{table}
}	%% small no tame
%}

%\vspace{5mm}

�ޤ���X ������ɥ��ˤ����륢�ץꥱ�������ˤ�
ɽ\ref{X������ɥ��ΤߤΥ�������}��
���Ҥ�����Ĥ����Τ⤢��ޤ�
(TTY�١��������ܸ����ϤǤϡ������Υ������Ҥ�̵�뤵��ޤ�)��
\HIDXAS{X ������ɥ��Υ���}{�ؤ�����Ȥ��Τ���}{H0.4.5.3.0}

%{\tt
{\small
\begin{table}[hbtp]
\begin{center}
\caption{X������ɥ��ΤߤΥ�������}\label{X������ɥ��ΤߤΥ�������}
\begin{tabular}{|l|l|}
\hline
\multicolumn{1}{|c|}{������} & \multicolumn{1}{|c|}{̾������} \\
\hline
CTRL+NFER & \tt "$\backslash$C-Nfer" \\
CTRL+XFER & \tt "$\backslash$C-Xfer" \\
CTRL+��   & \tt "$\backslash$C-Up" \\
CTRL+��   & \tt "$\backslash$C-Left" \\
CTRL+��   & \tt "$\backslash$C-Right" \\
CTRL+��   & \tt "$\backslash$C-Down" \\
SHIFT+NFER& \tt "$\backslash$S-Nfer" \\
SHIFT+XFER& \tt "$\backslash$S-Xfer" \\
SHIFT+��  & \tt "$\backslash$S-Up" \\
SHIFT+��  & \tt "$\backslash$S-Left" \\
SHIFT+��  & \tt "$\backslash$S-Right" \\
SHIFT+��  & \tt "$\backslash$S-Down" \\
\hline
\end{tabular}
\end{center}
\end{table}
}	%% small no tame
%}

%\vspace{5mm}

ɽ\ref{�㻲�͡������ʥ�������}�Υ����Ϥۤ��Υ����������ʾ�礬����ޤ���
�ǡ��ɤ��餫�����Υ����˵�ǽ�������Ƥ�ȡ�
���Υ�����\HIDX{�����ʥ���}{�Ȥ����ʤ���}{H0.4.5.3.0}�ˤ�
��ǽ�������Ƥ����Ȥˤʤ�ޤ���

%\vspace{5mm}

%{\tt
\begin{table}[hbtp]
\begin{center}
\caption{�㻲�͡������ʥ�������}\label{�㻲�͡������ʥ�������}
\begin{tabular}{|ll|l|}
\hline
\multicolumn{2}{|c|}{������} & \multicolumn{1}{|c|}{�� �� �� �� ��} \\
\hline
\tt CTRL+h & \tt (C-h) &  BS(Backspace) \\
\tt CTRL+i & \tt (C-i) &  TAB(Tab) \\
\tt CTRL+k & \tt (C-k) &  HOMECLR(Clear) \\
\tt CTRL+m & \tt (C-m) &  ENTER(Enter), RETURN(Return) \\
\tt CTRL+[ & \tt (C-[) &  ESC(Escape) \\
\tt CTRL+@ & \tt (C-@) &  CTRL+SPACE(C+Space) \\
\hline
\end{tabular}
\end{center}
\end{table}
%}

%\vspace{5mm}


\HIDX{�⡼��̾}{�⡼�Ȥᤤ}{H0.4.5.3.0}��
ɽ\ref{����������ƤΤǤ���⡼�ɰ���}�˼�����Τ�����ޤ���

%\vspace{5mm}

%{\tt
\begin{table}[hbtp]
\begin{center}
\caption{����������ƤΤǤ���⡼�ɰ���}\label{����������ƤΤǤ���⡼�ɰ���}
\begin{tabular}{|l|l|}
\hline
\multicolumn{1}{|c|}{̾������} & \multicolumn{1}{|c|}{�⡡��������������} \\
\hline
\HIDX{alpha-mode}{���̣Уȣ��ݣͣϣģ�}{H0.4.5.3.0}
		& ����ե��٥åȤ����Ϥ��Ƥ������  \\ \hline
\HIDX{empty-mode}{�ţͣУԣ١ݣͣϣģ�}{H0.4.5.3.0}
		& ���ܸ����ϥ⡼�ɤǤޤ�ʸ�������Ϥ��Ƥ��ʤ�����\\ \hline
\HIDX{yomi-mode}{�٣ϣͣϡݣͣϣģ�}{H0.4.5.3.0}
		& �ɤߤ����Ϥ��Ƥ������\\ \hline
\HIDX{mojishu-mode}{�ͣϣʣɣӣȣաݣͣϣģ�}{H0.4.5.3.0}
		& ʸ�����Ѵ���ԤäƤ������\\ \hline
\HIDX{tankouho-mode}{�ԣ��Σˣϣգȣϡݣͣϣģ�}{H0.4.5.3.0}
		& �Ѵ������򲡤��Ƹ����ɽ�����Ƥ������\\ \hline
\HIDX{ichiran-mode}{�ɣãȣɣң��Ρݣͣϣģ�}{H0.4.5.3.0}
		& �������ɽ���򤷤Ƥ������\\ \hline
\HIDX{henkan-nyuuryoku-mode}{�ȣţΣˣ��ΡݣΣ٣գգң٣ϣˣաݣͣϣģ�}{H0.4.5.3.0}
		& empty-mode��yomi-mode������\\ \hline 
\HIDX{kigou-mode}{�ˣɣǣϣաݣͣϣģ�}{H0.4.5.3.0}
		& ������������Ϥ��Ƥ������\\ \hline
\HIDX{yes-no-mode}{�٣ţӡݣΣϡݣͣϣģ�}{H0.4.5.3.0}
		& YES/NO���䤤��碌�����򤷤Ƥ������\\ \hline
\HIDX{on-off-mode}{�ϣΡݣϣƣơݣͣϣģ�}{H0.4.5.3.0}
		& ����ޥ���ȥ���ޥ���Ȥ���ꤷ�Ƥ������\\ \hline
\HIDX{shinshuku-mode}{�ӣȣɣΣӣȣգˣաݣͣϣģ�}{H0.4.5.3.0}
		& ʸ��򿭤Ф��̤ᤷ�Ƥ������\\ \hline
\end{tabular}
\end{center}
\end{table}
%}

%\vspace{5mm}

�⡼��̾�ϡ���ʸ������ʸ������̤��ޤ��������դ���������

\subsection{ʣ�������ǤΥ����γ������(\HIDX{�ޥ����������}{�ޤ��
��������}{H0.4.5.4.0})}
\label{4.5.5ʣ�������ǤΥ����γ������}

{\dg\bf \ref{4.5.2��ǽ�ؤΥ����γ������} \hpref{count=11,label=H0.4.5.3.0}
��ǽ�ؤΥ����γ������} �Ǥϡ�
1 �Ĥε�ǽ�� 1 �ĤΥ������������Ƥ���ˡ���������ޤ�����
�������������γ�����Ƥϥ����ȵ�ǽ�� 1 �� 1 �ˤ����б�����������ǤϤʤ���
¿��¿�б������뤳�Ȥ�Ǥ��ޤ���
���ʤ����ʣ���ε�ǽ��ʣ���Υ������˳�����Ƥ뤳�Ȥ��Ǥ��ޤ���
���Τ褦�ʵ��Ҥ򤹤뤳�Ȥˤ�������Ƥ��ޤ���

\begin{nquote}{3em}
\begin{verbatim}
(set-key  '�⡼��̾  "����1 ����2 �� ����n "
                            '(��ǽ̾1 ��ǽ̾2 �� ��ǽ̾n))
\end{verbatim}
\end{nquote}

\begin{nquote}{3em}
\begin{verbatim}
(global-set-key  "����1 ����2 �� ����n "
                            '(��ǽ̾1 ��ǽ̾2 �� ��ǽ̾n))
\end{verbatim}
\end{nquote}

\begin{nquote}{3em}
\begin{tabular}{ll}
(��)  & \\
 & ;C-x���ڡ����Ȳ����줿�Ȥ������ʴ����Ѵ�����\\
 & ;��ü��ʸ�᤬ȿž�����褦�ˤ��롣 \\
 & \verb!(set-key 'yomi-mode "\C-x\Space"! \\
 & \verb!��������������������������'(henken end-of-line)) !
\end{tabular}
\end{nquote}

������������������ˤ��ξ��֤���ȴ������� quit(�����)��
������Ƥ������򲡤��Ƥ���������
�ǥե���ȤǤ� \CTRL + \fbox{g} �Ǥ���

\subsection{\HIDX{�����γ�����Ʋ��}{�����Τ�ꤢ�Ƥ�������}{H0.4.5.5.0}}

���뵡ǽ���Ф��륭���γ�����Ƥ�̤������֤��᤹�Ȥ��ϡ�
global-unbind-key-function �ޤ��� unbind-key-function ��
�Ѥ��ơ��ʲ��Τ褦�˵��Ҥ��ޤ���

\HIDXAS{global-unbind-key-function}{�ǣ̣ϣ£��̡ݣգΣ£ɣΣġݣˣţ١ݣƣգΣãԣɣϣ�}{H0.4.5.5.0}
\begin{nquote}{3em}
\begin{verbatim}
(global-unbind-key-function '��ǽ̾)
\end{verbatim}
\end{nquote}

\HIDXAS{unbind-key-function}{�գΣ£ɣΣġݣˣţ١ݣƣգΣãԣɣϣ�}{H0.4.5.5.0}
\begin{nquote}{3em}
\begin{verbatim}
(unbind-key-function '�⡼��̾ '��ǽ̾)
\end{verbatim}
\end{nquote}

\vspace{5mm}

\begin{nquote}{3em}
\begin{tabular}{ll}
(��)  & \\
      & ;����Ū�ˡּ����׵�ǽ��̤����ˤ��� \\
      & \verb!(global-unbind-key-function 'quit)! \\
      & ;�ɤߥ⡼�ɤǡ��Ѵ��׵�ǽ��̤����ˤ��� \\
      & \verb!(unbind-key-function 'yomi-mode 'henkan)!
\end{tabular}
\end{nquote}

% ------------------------------ 3.3.6
\section{���Τۤ��Υ������ޥ���}
\label{3.3.6����¾�Υ������ޥ���}

�����ǡ��������ޥ����ե�����(.canna)�� \HIDX{���󥿥å�����§}
{���󤿤ä���������}{H0.4.6.0.0}�䡢
�������ޥ����ե�������Ѥ�����ǡ����ˤĤ����������ޤ���

% ------------------------------ 3.3.6.1
\subsection{\HIDX{���󥿥å����δ���}{���󤿤Ĥ����Τ��ۤ�}{H0.4.6.1.0}}
\label{3.3.6.1���󥿥å����δ���}

.canna �Υ��󥿥å����� Lisp ��Ʊ����ΤǤ���
���ʤ������ (�פȡ�) �פǰϤޤ줿��ʬ�� 1 �ĤΤ��Ȥ����ɽ�����Ƥ��ޤ���

�̾�ϡ��� (�פȡ�) �פǰϤޤ줿��ΤΤ�������ü��
���Ҥ���Ƥ����Τ��ؿ��ȤʤäƤ��ޤ���

\begin{CODEBOX}
(��)  \\
����(set-mode-display 'alpha-mode "--") \\
����set-mode-display �ϴؿ��Ǥ��� \\
\end{CODEBOX}

����ʳ��Τ�Τǡ� (�פȡ�) �פδ֤����äƤ����Τϴؿ��ΰ����Ǥ���

\begin{CODEBOX}
(��)  \\
����(set-mode-display 'alpha-mode "--") \\
����'alpha-mode, "-" �ϰ����Ǥ��� \\
\end{CODEBOX}

��������ʬ�ˤ���ˡ� (�פȡ�) �פǰϤޤ줿����񤯤��Ȥ��ǽ�Ǥ���

\begin{CODEBOX}
(��)  \\
���� (set-mode-display 'alpha-mode \\
������������������������(if use-default "  " "--")) \\
use-default�Ȥ����ѿ����ͤο����ͤˤ�ä�alpha-mode�Υ⡼�ɤΥ⡼��ɽ��ʸ������ڤ��ؤ��Ƥ��ޤ��� \\
\end{CODEBOX}

�����ϰ��١�\HIDX{ɾ��}{�Ҥ褦��}{H0.4.6.1.0}��
(\HIDX{evaluate}{�ţ֣��̣գ��ԣ�}{H0.4.6.1.0})����Ƥ���
�ؿ����Ϥ���ޤ����������Ĥ����㳰�⤢��ޤ���
���Ȥ��С�setq, if, defmode �ʤɤδؿ�(���Τˤϴؿ��ȤϸƤФʤ� Lisp ������
��¿��)�Ǥϡ����٤Ƥΰ�������ɾ���פ����櫓�ǤϤ���ޤ���

�����ǡ�ɾ���פȤϰ��٤��μ����¹Ԥ���Ƥ��μ����ͤ���뤳�Ȥ�ؤ��ޤ���

% ------------------------------ 3.3.6.2
\subsection{���ޤ��ޤʥǡ���}
\label{3.3.6.2���ޤ��ޤʥǡ���}

.canna�Ǥ�ɽ\ref{.canna�ˤ�����ɾ����}�˼������Υǡ������о줷�ޤ���
\HIDXAS{.canna�ˤ�����ɾ����}{��CANNA�ˤ�����Ҥ礦������}{H0.4.6.2.0}
%\vspace{5mm}

%{\tt
\begin{table}[hbtp]
\begin{center}
\caption{.canna�ˤ�����\HIDX{ɾ����}{�Ҥ褦������}{H0.4.6.2.0}}
\label{.canna�ˤ�����ɾ����}
\begin{tabular}{|l|l|}
\hline
\multicolumn{1}{|c|}{�ǡ�����} & \multicolumn{1}{|c|}{��} \\
\hline
������        & t, nil \\
������          & 1, 2, 3, �ġ� \\
ʸ����          & ?a, ?$\backslash$C-a, ?$\backslash$Xfer, �ġ� \\
ʸ����        & "romaji.kp", "fuzokugo", "[ �� ]", �ġ� \\
����ܥ�(����) & forward, next, alpha-mode, �ġ� \\
�ꥹ��        & (henkan end-of-line), (katakana kakutei), �� \\
\hline
\end{tabular}
\end{center}
\end{table}
%}

%\vspace{5mm}

�äˡ������ͤ� \HIDX{t}{��}{H0.4.6.2.0}(��)
��\HIDX{nil}{�Σɣ�}{H0.4.6.2.0}(��)��ɽ�����뤳�Ȥˤ����դ���������

% ------------------------------ 3.3.6.3
\subsection{\HIDX{�ѿ�}{�ؤ󤹤�}{H0.4.6.3.0}}
\label{3.3.6.3�ѿ�}

�ѿ��ϥ���ܥ��ɽ������ޤ���

�ѿ����ͤ��������������\HIDX{setq}{�ӣţԣ�}{H0.4.6.3.0}�ǹԤ��ޤ���

\vspace{5mm}

\begin{CODEBOX}
(��) \\
����(setq bunsetsu-kugiri  t) \\
�����ѿ�bunsetsu-kugiri�˿����������ޤ��� \\
����(setq n-henkan-for-ichiran  5) \\
�����ѿ�n-henkan-for-ichiran��5���������ޤ��� \\
\end{CODEBOX}

�����ѿ��򻲾Ȥ���ˤ��ѿ����Ȥ�ؿ��ΰ����˵��Ҥ��ޤ���

\begin{CODEBOX}
(��) \\
����(if bunsetsu-kugiri (setq select-direct nil)) \\
������ʸ����ڤ�פ����ˤʤäƤ����select-direct�򵶤ˤ��ޤ��� \\
\end{CODEBOX}

����ܥ�ϴؿ��ΰ�����Ϳ����줿�Ȥ����ѿ��Ȥ��ơ�ɾ���פ��졢
�ѿ��Ȥ��Ƥ����Ƥ������ޤ�������ܥ뼫�Ȥ��ͤȤ���Ϳ���������ϡ�
��'��(���󥰥륯������)�򥷥�ܥ�����ˤĤ��ޤ���

\vspace{5mm}

{\tt
\begin{center}
\begin{tabular}{|p{1cm}p{13cm}|}
\hline
(��) & \\
 & (global-set-key "$\backslash$Space" 'self-insert) \\
 & ���ڡ����������Ѵ��ǤϤʤ����ϤȤ��Ƽ�갷���ޤ������ξ�� \\
 & self-insert�Ȥ�������ܥ뼫�Ȥ�global-set-key�Ȥ����ؿ���Ϳ��������
���Τǡ� 
self-insert���Ф��ơ�'�פ��դ��Ƥ��ޤ��� \\
\hline
\end{tabular}
\end{center}
}

\vspace{5mm}

����ܥ�ʳ��Υǡ����Ρ�ɾ���׸���ͤ�ɽ\ref{����ܥ�ʳ��Υǡ�����}�˼����ޤ���

%\vspace{5mm}

{\tt
\begin{table}[hbtp]
\begin{center}
\caption{����ܥ�ʳ��Υǡ�����}\label{����ܥ�ʳ��Υǡ�����}
\begin{tabular}{|l|l|}
\hline
\multicolumn{1}{|c|}{�ǡ�����} & \multicolumn{1}{|c|}{ɾ�������} \\
\hline
������        & �����ͼ��� \\
����          & �����ǡ������� \\
ʸ��          & ʸ���ǡ������� \\
ʸ����        & ʸ����ǡ������� \\
����ܥ�(����)& �ѿ������ \\
�ꥹ��        & �ؿ���¹Ԥ����꥿������ \\
\hline
\end{tabular}
\end{center}
\end{table}
}

%\vspace{5mm}

ɽ\ref{����ܥ�ʳ��Υǡ�����}����狼��褦�ˡ�
����ܥ�ȥꥹ�Ȥ˴ؤ��ƤϤ����Υǡ������Ȥ�����ˤ���������
��'�פ�ɬ�פˤʤ�ޤ���
������������ʳ��Υǡ����ξ��ϡ�'�פ�Ĥ��ʤ��Ȥ⹽���ޤ���

�ɤΤ褦��̾���Υ���ܥ�Ǥ��ѿ��Ȥ����Ѥ��뤳�Ȥ��Ǥ��ޤ�������������Ԥ�
�ʤ����ѿ��λ��Ȥ�Ԥ��ȡ�Unbound Variable�ץ��顼�ˤʤ�ޤ���

{\dg\bf \ref{3.3.6.4�������ޥ����Υ������} 
(\ref{�������ޥ���������¾�Υ������}) \hpref{count=10,label=H0.4.6.4.0.100}���Τۤ��Υ������} �˵��Ҥ���Ƥ����ѿ��ϡ�
���ʴ����Ѵ��Υ������ޥ���������ɤȤ��Ƥΰ�̣������ޤ���
���Τ褦���ѿ����ͤ������������ϡ����ʴ����Ѵ������˱ƶ���Ϳ���ޤ���

% ------------------------------ 3.3.6.4
\subsection{�������ޥ����Υ������}
\label{3.3.6.4�������ޥ����Υ������}

�������ޥ����Τ����\HIDX{�������}{�������}{H0.4.6.4.0}��
���٤ƥ���ܥ�Ȥ��Ƥ�ɽ������äƤ��ޤ�����
���μ��֤ϴؿ��Ǥ��ä����ѿ��Ǥ��ä��ꤷ�ޤ����ʲ��˳��פ򼨤��ޤ���

\begin{enumerate}
% ------------------------------ (1)
\hptar{label=H0.4.6.4.0.1}
\item {\dg\bf \HIDX{���������}{������Τ�褦}
{H0.4.6.4.0} $-$ \HIDX{use-dictionary}{�գӣšݣģɣãԣɣϣΣ��ң�}
{H0.4.6.4.0}}

\begin{CODEBOX}
(use-dictionary "iroha" "fuzokugo") \\
\end{CODEBOX}

���Ѥ��뼭�����ꤷ�ޤ���

% misao �� ʸˡ�����ɲ� %
���󼭽�ʸˡ����ñ����Ͽ�Ѽ���ʤɤ���ꤹ��ˤϼ����
̾����ľ���� \HIDX{:bushu}{���£գӣȣ�}{H0.4.6.4.0} �� 
\HIDX{:grammar}{���ǣң��ͣͣ���}{H0.4.6.4.0}, 
\HIDX{:user}{���գӣţ�}{H0.4.6.4.0} ����Ԥ����ޤ���

\vspace{5mm}

\begin{CODEBOX}
(use-dictionary "iroha" "fuzokugo" :bushu "bushu" \\
����������������������������������������:grammar "mygram" ) \\
����������������������������������������:user "mydic" ) \\
\end{CODEBOX}

\begin{nquote}{1em}
\begin{namelist}{��}
\item[����] :bushu, :grammar, :user �ϥ���ܥ�ʤΤ�
����� ':bushu, ':user �Τ褦�ˡ�'��(���󥰥륯������)��Ĥ��ʤ����
�ʤ�ʤ��ΤǤ�����
��:��(������)����Ϥޤ륷��ܥ�(������ɥ���ܥ�Ȥ����ޤ�)�˸¤äƤ�
���פǤ���
\end{namelist}
\end{nquote}

�� (�פȡ�) �פδ֤�Ŭ���˲��Ԥ��Ƥ⹽���ޤ���

\vspace{5mm}
\begin{center}\tt
\begin{tabular}{|p{20em}|} \hline
(use-dictionary \\
"iroha" "fuzokugo" :bushu"bushu" :grammar "mygram" :user "mydic") \\ \hline
\end{tabular}
\end{center}
\vspace{5mm}


use-dictionary �˴ؤ��������� {\bf\dg \ref{3.3.2���Ѥ��뼭��λ���} 
\hpref{count=9,label=H0.4.2.0.0}���Ѥ��뼭��λ���} �ˤ⤢��ޤ�
�Τǻ��Ȥ��Ƥ���������

% ------------------------------ (2)
\hptar{label=H0.4.6.4.0.2}
\item {\dg\bf \HIDX{�⡼��ɽ�����ѹ�}{�⡼�ȤҤ礦���Τؤ󤳤�}
{H0.4.6.4.0.2} $-$ \HIDX{set-mode-display}{�ӣţԡݣͣϣģšݣģɣӣУ̣���}
{H0.4.6.4.0.2}}
\label{�������ޥ������⡼��ɽ�����ѹ�}

\begin{CODEBOX}
(set-mode-display 'henkan-nyuuryoku-mode "[�Ѵ�����]") \\
\end{CODEBOX}

���ϥ⡼�ɤ�ɽ��ʸ������ѹ����ޤ���
��̤ϼ��Τ褦�ˤʤ�ޤ���

\begin{SCREEN}
\verb+% + \CURSOR \\
{[�Ѵ�����]} \\
\end{SCREEN}

�⡼��ɽ��ʸ������ѹ����ʤ��ǡ����Υ⡼�ɤΥ⡼��ɽ��ʸ�����Ʊ��ɽ����
���������ϡ�ʸ���������� nil �򵭽Ҥ��ޤ���

����Ǥ���⡼�ɤ�ɽ\ref{�⡼�ɰ���}�˼����ޤ���
\HIDXAS{�⡼��̾}{�⡼�Ȥᤤ}{H0.4.6.4.0}

����ե��٥å����ϥ⡼��(alpha-mode)���Ѵ����ϥ⡼��(henkan-nyuuryoku-mode)��
�⡼��ɽ��ʸ����� nil �����ꤷ����硢���ץꥱ�������ˤ�äƤϡ�
�⡼��ɽ��ʸ����ɽ������ʤ��ʤ뤳�Ȥ�����ޤ���
����ե��٥å����ϥ⡼�ɡ��Ѵ����ϥ⡼�ɤǤ� nil �ε��Ҥ�
�򤱤������ɤ��Ǥ��礦��

���ޤ�Ĺ��ʸ�������ꤹ��ȡ����ץꥱ�������ˤ�äƤ�
ʸ����θ������ڤ��ɽ�����줿�ꡢ����ư����ꤹ�뤳�Ȥ�����ޤ���

\newpage
%\vspace{5mm}
\HIDXAS{�⡼�ɰ���}{�⡼�Ȥ������}{H0.4.6.4.0}
{\tt \small
\begin{table}[hbtp]
\begin{center}
\caption{�⡼�ɰ���}\label{�⡼�ɰ���}
\begin{tabular}{|l|c|l|}
\hline
\multicolumn{1}{|c|}{�ء���������} & \multicolumn{1}{|c|}{�ǥե����} & \multicolumn{1}{|c|}{�⡡��������������} \\
\hline
\HIDX{alpha-mode}{���̣Уȣ��ݣͣϣģ�}{H0.4.6.4.0}
		 &          &  ����ե��٥å����ϥ⡼�� \\
\hline
\HIDX{henkan-nyuuryoku-mode}{�ȣţΣˣ��ΡݣΣ٣գգң٣ϣˣաݣͣϣģ�}{H0.4.6.4.0} 
		 & [ �� ]   &  �Ѵ����ϥ⡼�� \\
\hline
\HIDX{kigou-mode}{�ˣɣǣϣաݣͣϣģ�}{H0.4.6.4.0}
		 & [����]   &  �������ɽ������ \\
\hline
\HIDX{yomi-mode}{�٣գͣɡݣͣϣģ�}{H0.4.6.4.0}
		 & nil      &  �ɤߤ����Ϥ��Ƥ������ \\
\hline
\HIDX{mojishu-mode}{�ͣϣʣɣӣȣաݣͣϣģ�}{H0.4.6.4.0}
		 & [����]   &  ʸ�����Ѵ����� \\
\hline
\HIDX{tankouho-mode}{�ԣ��Σˣϣգȣϡݣͣϣģ�}{H0.4.6.4.0}
		 & [����]   &  ñ����ɽ������ \\
\hline
\HIDX{ichiran-mode}{�ɣãȣɣң��Ρݣͣϣģ�}{H0.4.6.4.0}
		 & [����]   &  �������ɽ������ \\
\hline
\HIDX{yes-no-mode}{�٣ţӡݣΣϡݣͣϣģ�}{H0.4.6.4.0}
		 & [����]   &  ���ѼԤ˼���򤷤Ƥ������ \\
\hline
\HIDX{on-off-mode}{�ϣΡݣϣƣơݣͣϣģ�}{H0.4.6.4.0}
		 & nil      &  ����ޥ���ȥ���ޥ���Ⱦ��֤ʤ� \\
\hline
\HIDX{shinshuku-mode}{�ӣȣɣΣӣȣգˣաݣͣϣģ�}{H0.4.6.4.0}i
		 & [ʸ��]   &  ʸ�῭�̥⡼�� \\
\hline
\HIDX{chikuji-yomi-mode}{�ãȣɣˣգʣɡݣ٣ϣͣɡݣͣϣģ�}{H0.4.6.4.0}
		 & [�༡]   &  �༡��ư���ϻ����ɤ���ʬ \\
\hline
\HIDX{chikuji-bunsetsu-mode}{�ãȣɣˣգʣɡݣ£գΣӣţԣӣաݣͣϣģ�}{H0.4.6.4.0}
		 & [�༡]   &  �༡��ư���ϻ���ʸ����ʬ \\
\hline
\HIDX{zen-hira-henkan-mode}{�ڣţΡݣȣɣң��ݣȣţΣˣ��Ρݣͣϣģ�}{H0.4.6.4.0}
		 & [����]   &  ���ѤҤ餬�ʥ⡼�� \\
\hline
\HIDX{han-hira-henkan-mode}{�ȣ��Ρݣȣɣң��ݣȣţΣˣ��Ρݣͣϣģ�}{H0.4.6.4.0}
		 & [Ⱦ��]   &  Ⱦ�ѤҤ餬�ʥ⡼�� \\
\hline
\HIDX{zen-kata-henkan-mode}{�ڣţΡݣˣ��ԣ��ݣȣţΣˣ��Ρݣͣϣģ�}{H0.4.6.4.0}
		 & [����]   &  ���ѥ������ʥ⡼�� \\
\hline
\HIDX{han-kata-henkan-mode}{�ȣ��Ρݣˣ��ԣ��ݣȣţΣˣ��Ρݣͣϣģ�}{H0.4.6.4.0}
		 & [Ⱦ��]   &  Ⱦ�ѥ������ʥ⡼�� \\
\hline
\HIDX{zen-alpha-henkan-mode}{�ڣţΡݣ��̣Уȣ��ݣȣţΣˣ��Ρݣͣϣģ�}{H0.4.6.4.0}
		 & [����]   &  ���ѱѿ��⡼�� \\
\hline
\HIDX{han-alpha-henkan-mode}{�ȣ��Ρݣ��̣Уȣ��ݣˣţΣˣ��Ρݣͣϣģ�}{H0.4.6.4.0}
		 & [Ⱦ��]   &  Ⱦ�ѱѿ��⡼�� \\
\hline
\HIDX{zen-hira-kakutei-mode}{�ڣţΡݣȣɣң��ݣˣ��ˣգԣţɡݣͣϣģ�}{H0.4.6.4.0}
		 & $<$����$>$   &  ���ѤҤ餬�ʳ���⡼�� \\
\hline
\HIDX{han-hira-kakutei-mode}{�ȣ��Ρݣȣɣң��ݣˣ��ˣգԣţɡݣͣϣģ�}{H0.4.6.4.0}
		 & $<$Ⱦ��$>$   &  Ⱦ�ѤҤ餬�ʳ���⡼�� \\
\hline
\HIDX{zen-kata-kakutei-mode}{�ڣţΡݣˣ��ԣ��ݣˣ��ˣգԣţɡݣͣϣģ�}{H0.4.6.4.0}
		 & $<$����$>$   &  ���ѥ������ʳ���⡼�� \\
\hline
\HIDX{han-kata-kakutei-mode}{�ȣ��Ρݣˣ��ԣ��ݣˣ��ˣգԣţɡݣͣϣģ�}{H0.4.6.4.0}
		 & $<$Ⱦ��$>$   &  Ⱦ�ѥ������ʳ���⡼�� \\
\hline
\HIDX{zen-alpha-kakutei-mode}{�ڣţΡݣ��̣Уȣ��ݣˣ��ˣգԣţɡݣͣϣģ�}{H0.4.6.4.0}
		 & $<$����$>$   &  ���ѱѿ�����⡼�� \\
\hline
\HIDX{han-alpha-kakutei-mode}{�ȣ��Ρݣ��̣Уȣ��ݣˣ��ˣգԣţɡݣͣϣģ�}{H0.4.6.4.0}
		 & $<$Ⱦ��$>$   &  Ⱦ�ѱѿ�����⡼�� \\
\hline
\HIDX{hex-mode}{�ȣţءݣͣϣģ�}{H0.4.6.4.0}
		 & [16��]   &  ���������Ͼ��� \\
\hline
\HIDX{bushu-mode}{�£գӣȣաݣͣϣģ�}{H0.4.6.4.0}
		 & [����]   &  �������ϥ⡼�� \\
\hline
\HIDX{extend-mode}{�ţأԣţΣġݣͣϣģ�}{H0.4.6.4.0}
		 & [��ĥ]   &  �桼�ƥ���ƥ��⡼�� \\
\hline
\HIDX{russian-mode}{�ңգӣӣɣ��Ρݣͣϣģ�}{H0.4.6.4.0}
		 & [~��~]   &  ������ʸ���������� \\
\hline
\HIDX{greek-mode}{�ǣңţţˡݣͣϣģ�}{H0.4.6.4.0}
		 & [~��~]   &  ���ꥷ��ʸ���������� \\
\hline
\HIDX{line-mode}{�̣ɣΣšݣͣϣģ�}{H0.4.6.4.0}
		 & [����]   &  ������������ \\
\hline
\HIDX{changing-server-mode}{�ãȣ��ΣǣɣΣǡݣӣţң֣ţҡݣͣϣģ�}{H0.4.6.4.0}
		 & [�ѹ�]   &  �������ѹ����� \\
\hline
\HIDX{henkan-method-mode}{�ȣţΣˣ��Ρݣͣţԣȣϣġݣͣϣģ�}{H0.4.6.4.0}
		 & [�Ѵ�]   &  �Ѵ���ˡ�ѹ����� \\
\hline
\HIDX{delete-dic-mode}{�ģţ̣ţԣšݣģɣáݣͣϣģ�}{H0.4.6.4.0}
		 & [���]   &  ������ \\
\hline
\HIDX{touroku-mode}{�ԣϣգңϣˣաݣͣϣģ�}{H0.4.6.4.0}
		 & [��Ͽ]   &  ñ����Ͽ���� \\
\hline
\HIDX{touroku-hinshi-mode}{�ԣϣգңϣˣաݣȣɣΣӣȣɡݣͣϣģ�}{H0.4.6.4.0}
		 & [�ʻ�]   &  �ʻ�������� \\
\hline
\HIDX{touroku-dic-mode}{�ԣϣգңϣˣաݣģɣáݣͣϣģ�}{H0.4.6.4.0}
		 & [����]   &  ����������� \\
\hline
\HIDX{quoted-insert-mode}{�ѣգϣԣţġݣɣΣӣţңԡݣͣϣģ�}{H0.4.6.4.0}
		 & [~��~]   &  �������Ͼ��� \\
\hline
\HIDX{mount-dic-mode}{�ͣϣգΣԡݣģɣáݣͣϣģ�}{H0.4.6.4.0}
		 & [����]   &  ����ޥ���ȡ�����ޥ����������� \\
\hline
\end{tabular}
\end{center}
\end{table}
}

%\vspace{5mm}

%\newpage
% ------------------------------ (3)
\hptar{label=H0.4.6.4.0.3}
\item {\dg\bf \HIDX{�����ΥХ����}{�����ΤϤ����}
{H0.4.6.4.0} $-$ \HIDX{set-key}{�ӣţԡݣˣţ�}{H0.4.6.4.0},
\HIDX{global-set-key}{�ǣ̣ϣ£��̡ݣӣţԡݣˣţ�}{H0.4.6.4.0}}
\label{�������ޥ����������ΥХ����}

%\vspace{5mm}

\begin{CODEBOX}
(set-key  '�⡼��̾  "������"  '��ǽ��) \\
(global-set-key  "������"  '��ǽ��) \\
\end{CODEBOX}

set-key �Ǥϡ֥⡼��̾�פǻ��ꤵ�줿�⡼�ɤ��Ф��ƤΤߥ����Х���ɤ��ޤ���
global-set-key �Ϥ��٤ƤΥ⡼�ɤ��Ф���Ʊ�ͤ˥�����Х���ɤ��ޤ���

������ˤ� 1 ʸ���ʾ��ʸ���򵭽Ҥ��ޤ��������� {\tt "a", "B"} �ʤɤȽ�
����ۤ�������ȥ����륭���� {\tt "$\backslash$C-a"} �Τ褦�� {\tt
$\backslash$C-} ����Ԥ����Ƶ��Ҥ��뤳�Ȥ��Ǥ��ޤ���C-x �򲡤��Ƥ��� C-a ��
�����������Ф��ƥ����������Ƥ�Ȥ����褦�� C-x, C-a �Ȥ�����������
����ɽ������Ȥ��� {\tt "$\backslash$C-x$\backslash$C-a"} �ȵ��Ҥ��ޤ���

Xfer �ʤ��ü�ʥ����˴ؤ��Ƥϰʲ��Τ褦�˵��Ҥ��ޤ���
\HIDXAS{�ü쥭��}{�Ȥ����椭��}{H0.4.6.4.0}

\vspace{5mm}

{\tt \small
\begin{center}
\begin{tabular}{|l|}
\hline
\verb+ "\Space     "\Escape"   "\Tab"      "\Nfer"     "\Xfer"     "\Backspace" + \\
\verb+ "\Delete"   "\Insert"   "\Rollup"   "\Rolldown" "\Up"       "\Left" + \\
\verb+ "\Right"    "\Down"     "\Home"     "\Clear"    "\Help"     "\Enter" + \\
\verb+ "\Return"   "\F1"       "\F2"       "\F3"       "\F4"       "\F5" + \\
\verb+ "\F6"       "\F7"       "\F8"       "\F9"       "\F10"      "\Pf1" + \\
\verb+ "\Pf2"      "\Pf3"      "\Pf4"      "\Pf5"      "\Pf6"      "\Pf7" + \\
\verb+ "\Pf8"      "\Pf9"      "\Pf10"     "\S-Nfer"   "\S-Xfer"   "\S-Up" + \\
\verb+ "\S-Down"   "\S-Left"   "\S-Right"  "\C-Nfer"   "\C-Xfer"   "\C-Up" + \\
\verb+ "\C-Down"   "\C-Left"   "\C-Right" + \\
\hline
\end{tabular}
\end{center}
}

\vspace{5mm}

��ǽ��ˤϡ�ñ��ε�ǽ�򵭽Ҥ��뤫ʣ���ε�ǽ��� (�פȡ�) �פǰϤ�ǻ�
�ꤷ�ޤ���ʣ���ε�ǽ����ꤷ�����ϡ����ꤷ����ǽ�����ꤷ����˼¹Ԥ���ޤ���
����Ǥ��뵡ǽ̾�ΰ�����
{\dg\bf ��Ͽ\ref{C�������ޥ������Ѥ��뵡ǽ̾����ɽ} \hpref{count=16,label=H0.C.0.0.0}�������ޥ������Ѥ��뵡ǽ̾����ɽ} �򻲾Ȥ��Ƥ���������

set-key, global-set-key �ˤĤ��Ƥϡ�{\dg\bf\ref{3.3.5�������Υ�������
����} \hpref{count=11,label=H0.4.5.0.0}�������Υ������ޥ���} �ˤ�
���Ҥ�����ޤ��Τǻ��Ȥ��Ƥ���������

% ------------------------------ (4)
\hptar{label=H0.4.6.4.0.4}
\item {\bf\dg \HIDX{�����ȵ�ǽ�Υ���Х����}{�����Ȥ��Τ��Τ���Ϥ����}
{H0.4.6.4.0} $-$ \HIDX{unbind-key-function}
{�գΣ£ɣΣġݣˣţ١ݣƣգΣãԣɣϣ�}{H0.4.6.4.0}, 
\HIDX{global-unbind-key-function}
{�ǣ̣ϣ£��̡ݣգΣ£ɣΣġݣˣţ١ݣƣգΣãԣɣϣ�}{H0.4.6.4.0}}

%\vspace{5mm}

\begin{CODEBOX}
(unbind-key-function '�⡼��̾ '��ǽ̾) \\
(global-unbind-key-function '��ǽ̾) \\
\end{CODEBOX}

unbind-key-function �Ǥ�
�֥⡼��̾�פǻ��ꤵ�줿�⡼�ɤ��Ф��ƤΤߥ����򥢥�Х���ɤ��ޤ���
global-unbind-key-function �ǤϤ��٤ƤΥ⡼�ɤ��Ф��ƥ����򥢥�Х���ɤ��ޤ���

��ǽ̾�ˤϵ�ǽ�ꥹ�Ȥ򵭽Ҥ��뤳�ȤϤǤ��ޤ���

% ------------------------------ (5)
\hptar{label=H0.4.6.4.0.5}
\item {\dg\bf \HIDX{�⡼�����}{�⡼�ȤƤ���}{H0.4.6.4.0.5} $-$ 
\HIDX{defmode}{�ģţƣͣϣģ�}{H0.4.6.4.0.5}}
\label{�������ޥ������⡼�����}

%\vspace{5mm}

\begin{CODEBOX}
(defmode �⡼�ɥ���ܥ� "�⡼��ɽ��" "�����޻������Ѵ��ơ��֥�" \\
������������������������'(��ǽ�ꥹ��) �������ѥե饰)\\
\end{CODEBOX}

�������⡼�ɤ�������ޤ���

�������⡼�ɤǤϡ������Х�����ѤΤ��Υ⡼���ȼ��Υ����ޥåץơ��֥����
���Υ����޻������Ѵ��ơ��֥���ȼ��Υ⡼��ɽ��ʸ����ʤɤ������Ƥ��ޤ���

�⡼�ɥ���ܥ�ˤϡ����Υ⡼�ɤ򻲾Ȥ���Ȥ��Τ���Υ���ܥ����ꤷ�ޤ���
�ʲ����Υ⡼�ɤ򻲾Ȥ���Ȥ��ˤϤ��Υ⡼�ɥ���ܥ�ǻ��Ȥ���ޤ����⡼��
����ܥ��ɾ�����줺�Ѥ����ޤ����������äơ�'�פ�Ĥ���ɬ�פϤ����
���󡣥⡼�ɥ���ܥ��ɬ�����ꤷ�Ƥ���������

���Υ���ܥ�ϼ��Τ褦�����ѤǤ��ޤ���

\begin{enumerate}
% ------------------------------ (a)
\item �֤��Υ⡼�ɤ˰ܹԤ���פȤ�����ǽ��ɽ����ǽ̾

\begin{CODEBOX}
(��) \\
����(set-key 'henkan-nyuuryoku-mode "\verb!\!C-t" \\
����������������������������������'������줿����ܥ�) \\
\end{CODEBOX}

% ------------------------------- (b)
\item ���Υ⡼�ɤǤΤߥ����������Ƥ��ꡢ���Υ⡼�ɤ��б�����⡼��ʸ��������ꤷ���ꤹ��Ȥ��Υ⡼��̾

\end{enumerate}

\begin{CODEBOX}
(��) \\
����(set-mode-display '������줿����ܥ� \\
����������������������������������"[���⡼��]") \\
����(set-key '������줿����ܥ� \\
������������"\verb!\!C-t" 'henkan-nyuuryoku-mode) \\
\end{CODEBOX}

�⡼�ɥ���ܥ�ʳ��ΰ����Ͼ�ά��ǽ�Ǥ���
��ά�������� nil �����ꤵ�줿�Τ�Ʊ���������Ԥ��ޤ���

��ǽ�ꥹ�Ȥˤϡ�katakana, hiragana, romaji, kakutei, zenkaku, hankaku �ʤɤ�
����Ǥ��ޤ��������ϥ����޻������Ѵ���ȼ�äƼ¹Ԥ���뵡ǽ��ɽ���Ƥ��ޤ���

�Ǹ�ΰ����ֵ������ѥե饰�פˤϡ�
defsymbol�ˤ������� defmode �Ǻ��������⡼�ɤǻ��Ѥ��뤫�ɤ�������ꤷ�ޤ�
(defsymbol �ˤĤ��Ƥϡ�
{\dg\bf (\ref{�������ޥ������������}) \hpref{count=4,label=H0.4.6.4.0.6}�������} �򻲾Ȥ��Ƥ�������)��

\begin{CODEBOX}
(��) \\
����;�����������ϥ⡼�ɤ�������ޤ��� \\
����(defmode katakana-mode "[����]" romkana-table \\
������������������������������'(katakana) t) \\
����;C-k�ǥȥ��뤹��褦�ˤ��ޤ��� \\
����(set-key 'henkan-nyuuryoku-mode "\verb!\!C-k" \\
������������������������������'katakana-mode) \\
����(set-key 'katakana-mode "\verb!\!C-k" \\
������������������������������'henkan-nyuuryoku-mode) \\
����;�������ϥ⡼��(����Ū��)��������ޤ��� \\
����(defmode kana-input-mode "[����]" "kana.kp") \\
����;C-r�ǥȥ��뤹��褦�ˤ��ޤ��� \\
����(set-key 'henkan-nyuuryoku-mode "\verb!\!C-r" \\
������������������������������'kana-input-mode) \\
����(set-key 'kana-input-mode "\verb!\!C-r" \\
������������������������������'henkan-nyuuryoku-mode) \\
\end{CODEBOX}

% ------------------------------ (6)
\hptar{label=H0.4.6.4.0.6}
\item {\dg\bf \HIDX{�������}{�������Ƥ���}{H0.4.6.4.0} $-$ 
\HIDX{defsymbol}{�ģţƣӣ٣ͣ£ϣ�}{H0.4.6.4.0}}
\label{�������ޥ������������}

\begin{CODEBOX}
(defsymbol ?[ "��" "��" ) \\
\end{CODEBOX}

defsymbol �ϥ����޻������Ѵ����亴���뵬§�򵭽Ҥ��ޤ���
�����޻������Ѵ���§�ˤ����Ƥ����ϤΥ������Ѵ������
�֤��ʡפȤε�§�� 1 �� 1 �Ǥ����б��Ǥ��ޤ���Ǥ�������
defsymbol ���������뵬§�Ǥ� 1 �ĤΥ������Ф���
�����Ĥ�θ����������뤳�Ȥ��Ǥ��ޤ���

������ {\tt"�� "} ���Ф��� {\tt"�� "} �� {\tt"�� "} ���Ѵ�����
�Ȥ����롼��򵭽Ҥ�����Ǥ���

���Τ褦�ˤ��뤳�Ȥˤ�ꡢ�̾�� {\tt"�� "} ���Ф��� {\tt"�� "} ��
���Ϥ���ޤ�����{\tt"�� "} ���������Ϥ��Ƥ�����֤�
�Ѵ��������ǤĤ��Ȥˤ�� {\tt"�� "} �� {\tt"�� "} ��
�Ѵ�����褦���ڤ��ؤ��뤳�Ȥ��Ǥ��ޤ���
�ҤȤ����ڤ��ؤ�����ȡ�
������� {\tt"�� "} ������Ū�����Ϥ����褦�ˤʤ�ޤ���

defsymbol ��ʣ���󵭽Ҥ򤹤뤳�Ȥˤ�ꡢ�ڤ��ؤ����󥯤��뤳�Ȥ��Ǥ��ޤ���

\begin{CODEBOX}
(��) \\
\verb!����(defsymbol ?[ "��" "�� "! \\
\verb!�������������� ��?] "��" "�� " )! \\
\end{CODEBOX}

���Τ褦��������뤳�Ȥˤ�� {\tt"��"} �� {\tt"��"} ���ڤ��ؤ��ȡ���ưŪ�� 
{\tt"��"} �� {\tt" ��"} ���ڤ��ؤ��ޤ���

% ------------------------------ (7)
\hptar{label=H0.4.6.4.0.7}
\item {\dg\bf \HIDX{��˥塼�����}{��ˤ桼�ΤƤ���}{H0.4.6.4.0} $-$ 
\HIDX{defmenu}{�ģţƣͣţΣ�}{H0.4.6.4.0}}

\begin{CODEBOX}
(defmenu ��˥塼̾ \\
����("ɽ��ʸ����1" ��ǽ����ܥ�1) \\
����("ɽ��ʸ����2" ��ǽ����ܥ�2) \\
������������\\
) \\
\end{CODEBOX}

��������˥塼��������ޤ���

��˥塼̾�˻��ꤷ��̾���ϡ����Υ�˥塼��ƤӽФ��Ȥ��ε�ǽ̾�Ȥ��ƻ�
����褦�ˤʤ�ȤȤ�ˡ�set-mode-display �ǥ⡼��ɽ��ʸ������ѹ�����
�Ȥ��Υ⡼��̾�Ȥ��Ƥ�Ȥ���褦�ˤʤ�ޤ�(set-mode-display �ˤĤ��Ƥϡ�
{\dg\bf (\ref{�������ޥ������⡼��ɽ�����ѹ�}) \hpref{count=8,label=H0.4.6.4.0.2}�⡼��ɽ�����ѹ�} ��
���Ȥ��Ƥ�������)��

ɽ��ʸ����ˤϡ����Υ�˥塼�ΰ���ɽ����ɽ����������̾����ꤷ�ޤ���

��ǽ����ܥ�ˤϡ�ɽ\ref{��˥塼}�˼�����ǽ̾���ޤ��� defselection ���������
��ǽ����ܥ����ꤷ�ޤ�(defselection �ˤĤ��Ƥϡ�
{\dg\bf (\ref{�������ޥ�����ʸ�������κ���}) \hpref{count=7,label=H0.4.6.4.0.8}ʸ�������κ���} ��
���Ȥ��Ƥ�������)��
�ޤ�����ǽ����ܥ�ˡ��̤� defmenu ��������줿��˥塼̾����ꤹ�뤳�Ȥˤ��
��˥塼�ĥ꡼�������뤳�Ȥ�Ǥ��ޤ���

\begin{table}[hbtp]
\begin{center}

\caption{\dg\bf ��˥塼�˻���Ǥ��뵡ǽ̾����}\label{��˥塼}
\begin{tabular}{|l|l|}
\hline
\multicolumn{1}{|c|}{��  ǽ  ̾} & \multicolumn{1}{|c|}{��������������ǽ} \\
\hline
\HIDX{kigou-mode}{�ˣɣǣϣաݣͣϣģ�}{H0.4.6.4.0}
		 & �������̤ΰ����⡼�ɤˤʤ�\\ 
\hline
\HIDX{russian-mode}{�ңգӣӣɣ��Ρݣͣϣģ�}{H0.4.6.4.0}
		 & ������ʸ�������⡼�ɤˤʤ�\\
\hline
\HIDX{greek-mode}{�ǣңţţˡݣͣϣģ�}{H0.4.6.4.0}
		 & ���ꥷ��ʸ�������⡼�ɤˤʤ�\\
\hline
\HIDX{line-mode}{�̣ɣΣšݣͣϣģ�}{H0.4.6.4.0}
		 & ���������⡼�ɤˤʤ�\\
\hline
\HIDX{hex-mode}{�ȣţءݣͣϣģ�}{H0.4.6.4.0}
		 & 16 �ʥ��������ϥ⡼�ɤˤʤ�\\
\hline
\HIDX{bushu-mode}{�£գӣȣաݣͣϣģ�}{H0.4.6.4.0}
		 & �������ϥ⡼�ɤˤʤ�\\
\hline
\HIDX{touroku-mode}{�ԣϣգңϣˣաݣͣϣģ�}{H0.4.6.4.0}
		 & ñ����Ͽ�⡼�ɤˤʤ�\\
\hline
\HIDX{delete-dic-mode}{�ģţ̣ţԣšݣģɣáݣͣϣģ�}{H0.4.6.4.0}
		 & ñ�����⡼�ɤˤʤ�\\
\hline
\HIDX{jisho-ichiran}{�ʣɣӣȣϡݣɣãȣɣң���}{H0.4.6.4.0}
		 & ����ޥ���ȡ�����ޥ���Ȥ�Ԥ�\\
\hline
\HIDX{chikuji-mode}{�ãȣɣˣգʣɡݣͣϣģ�}{H0.4.6.4.0}
		 & �༡��ư�Ѵ����ڤ��ؤ���\\
\hline
\HIDX{renbun-mode}{�ңţΣ£գΡݣͣϣģ�}{H0.4.6.4.0}
		 & Ϣʸ���Ѵ����ڤ��ؤ���\\
\hline
\HIDX{disconnect-server}{�ģɣӣãϣΣΣţãԡݣӣţң֣ţ�}{H0.4.6.4.0}
		 & �����ФȤ���³���ڤ�\\
\hline
\HIDX{switch-server}{�ӣףɣԣãȡݣӣţң֣ţ�}{H0.4.6.4.0}
		 & �����Ф��ڤ��ؤ���Ԥ�\\
\hline
\HIDX{show-server-name}{�ӣȣϣסݣӣţң֣ţҡݣΣ��ͣ�}{H0.4.6.4.0}
		 & ������̾��ɽ������\\
\hline
\HIDX{show-gakushu}{�ӣȣϣסݣǣ��ˣգӣȣ�}{H0.4.6.4.0}
		 & �ؽ����֤�ɽ������\\
\hline
\HIDX{show-canna-version}{�ӣȣϣסݣã��ΣΣ��ݣ֣ţңӣɣϣ�}{H0.4.6.4.0}
		 & �С�������ɽ������\\
\hline
\HIDX{show-romkana-table}{�ӣȣϣסݣңϣͣˣ��Σ��ݣԣ��£̣�}{H0.4.6.4.0}
		 & �����޻������Ѵ��ơ��֥�̾��ɽ������\\
\hline
\HIDX{show-canna-file}{�ӣȣϣסݣã��ΣΣ��ݣƣɣ̣�}{H0.4.6.4.0}
		 & �������ޥ����ե�����̾��ɽ������\\
\hline
\HIDX{sync-dictionary}{�ӣ٣ΣáݣģɣãԣɣϣΣ��ң�}{H0.4.6.4.0}
		 & ����˽񤭹���\\
\hline
\end{tabular}
\end{center}
\end{table}

defmenu �����������˥塼̾��ǽ̾�Ȥ��ƻ��Ѥ���set-key �ʤɤǥ�����
�Х���ɤ��뤳�Ȥˤ�ꡢ��ʬ�Ǻ���������˥塼����Ѥ��뤳�Ȥ��Ǥ����
���ˤʤ�ޤ�(set-key �ˤĤ��Ƥϡ�
{\dg\bf (\ref{�������ޥ����������ΥХ����}) \hpref{count=7,label=H0.4.6.4.0.3}�����ΥХ����} ��
���Ȥ��Ƥ�������)��

\vspace{5mm}
%{\tt 
\begin{center}
\begin{tabular}{|ll|}
\hline
\verb+ (��) + & \\
\verb+ ����; ��˥塼�ĥ꡼�������� +   & \\
\verb+ ����; topmenu ����� +             & \\
\verb+ ����(defmenu  topmenu +            & \\
\verb+ ����  ("�������" kigou-mode) +    & \verb+ ; ��ǽ kigou-mode ��¹Ԥ��� + \\ 
\verb+ ����  ("��Ͽ" menu1) +             & \verb+ ; menu1 �� + \\
\verb+ ����) +                            & \\
\verb+  +                                 & \\
\verb+ ����; menu1 ����� +               & \\
\verb+ ����(defmenu menu1 +               & \\
\verb+ ����  ("��Ͽ" touroku-mode) +      & \verb+ ; ��ǽ touroku-mode ��¹Ԥ��� + \\
\verb+ ����  ("���" delete-dic-mode) +   & \verb+ ; ��ǽ delete-dic-mode ��¹Ԥ��� + \\
\verb+ ����  ("�������" jisho-ichiran) + & \verb+ ; ��ǽ jisho-ichiran ��¹Ԥ��� + \\
\verb+ ����) +                            & \\
\verb+   +                                & \\
\multicolumn{2}{|l|}{\verb+ ����(set-key 'empty-mode "+$\backslash$\verb+Help" 'topmenu) +}\\
\hline
\end{tabular}
\end{center}
%}
\vspace{5mm}

��˥塼�ĥ꡼��������ݡ��������ޥ����ν��֤�Ǥ�դǤ���
���ʤ�����嵭����Τ褦�� \verb+ (defmenu menu1 �� ) + �����ꤹ����
������ \verb+ (defmenu topmenu �� ) + ����� menu1 ����ꤹ�뤳��
����ǽ�Ǥ������������Ƶ�Ū�˥�˥塼�����ꤹ�뤳�ȤϤǤ��ޤ���

�������ޥ����ե�����γ�ĥ�⡼�ɥ�˥塼������˥��顼��¸�ߤ�����硢
���Υ�˥塼�����򤷤��Ȥ����ºݤˤϤ��Υ�˥塼�����򤵤줺��beep ��
���Ĥäơ��⡼�ɤϸ��ߤΥ⡼�ɤΤޤޤˤʤ�ޤ���


% misao �� defselectin ���ɲ� %
% ------------------------------ (8)
\hptar{label=H0.4.6.4.0.8}
\item {\dg\bf \HIDX{ʸ�������κ���}{�⤷�������Τ�������}{H0.4.6.4.0} $-$ 
\HIDX{defselection}{�ģţƣӣţ̣ţãԣɣϣ�}{H0.4.6.4.0}}
\label{�������ޥ�����ʸ�������κ���}

\begin{CODEBOX}
(defselection ��ǽ����ܥ� "ɽ��ʸ����" '(ʸ������)) \\
\end{CODEBOX}

ʸ��������������ޤ���

��ǽ����ܥ�ˤϡ����ε�ǽ�򻲾Ȥ���Ȥ��Τ���Υ���ܥ����ꤷ�ޤ���
�ʲ����ε�ǽ�򻲾Ȥ���Ȥ��ˤϡ����ε�ǽ����ܥ�ǻ��Ȥ���ޤ���

ɽ��ʸ����ˤϡ����ε�ǽ�˰ܹԤ����Ȥ��˥����ɥ饤���ɽ������
ʸ�������ꤷ�ޤ���ɽ��ʸ����Ͼ�ά���뤳�Ȥ�Ǥ��ޤ���
��ά������ϡ�ɽ��ʸ����� nil ����ꤷ�ޤ���
��ά���ϡ�ʸ���������ɽ������ľ����ɽ��ʸ����ɽ������ޤ���

ʸ�������ˤϡ�ɽ�������������򵭽Ҥ��ޤ���
ʸ�������λ�����ˡ�ϼ��� 2 �Ĥ���ˡ������ޤ���

\begin{nquote}{3em}
\begin{enumerate}
\item \label{string} �����Ǥ���֥륯�����ơ������ǰϤࡣ

\item \label{character} �����Ǥ����˥����������ޡ�����Ĥ��롣
\end{enumerate}
\end{nquote}

(\ref{string}) �ξ��ϡ������Ǥ� 2 ʸ���ʾ��ʸ����Ǥ���ꤹ�뤳�Ȥ��Ǥ��ޤ���

%\begin{CODEBOX}
\vspace{5mm}
\begin{center}\tt
\begin{tabular}{|p{14cm}|} \hline
(��) 1 ʸ������ɽ�������� \\
����;ʸ������ kakko ����� \\
����(defselection kakko "[���]" \\
������'("��" "��" "��" "��" "��" "��" "��" "��")) \\
(��) 2 ʸ������ɽ�������� \\
����;ʸ������ kakko ����� \\
����(defselection kakko "[���]" '("�ʡ�" "�̡�" "�Ρ�" "�С�")) \\
\hline
\end{tabular}
\end{center}
\vspace{5mm}
%\end{CODEBOX}

(\ref{character}) �ξ��ϡ�ʸ�����������Ǥ򤹤٤Ƶ��Ҥ���ΤǤϤʤ���
�ϰϻ��ꤹ�뤳�Ȥ�Ǥ��ޤ������������ϰϤ� EUC �����ɤΥ������ֹ��
����˻��ꤷ�Ƥ���������

%\begin{CODEBOX}
\vspace{5mm}
\begin{center}\tt
\begin{tabular}{|p{14cm}|} \hline
(��) ���Ǥ򤹤٤ƻ��ꤹ���� \\
����;ʸ������ keisan ����� \\
����(defselection keisan "[�׻�]" \\
������'(?�� ?�� ?�� ?�� ?�� ?�� ?�� ?�� ?�� ?�� ?�� ?�� ?��)) \\
(��) ���Ǥ��ϰϻ��ꤹ���� \\
����;ʸ������ keisan ����� \\
����(defselection keisan "[�׻�]" '(?�� - ?��)) \\
\hline
\end{tabular}
\end{center}
\vspace{5mm}
%\end{CODEBOX}


% ------------------------------ (9)
\hptar{label=H0.4.6.4.0.9}
\item {\dg\bf \HIDX{������֤�����}{���褭���褦�����Τ��ĤƤ�}
{H0.4.6.4.0} $-$ \HIDX{initialize-function}{�ɣΣɣԣɣ��̣ɣڣšݣ�
�գΣãԣɣϣ�}{H0.4.6.4.0}}


\begin{CODEBOX}
(initialize-function '(����������ꥷ������))\\
\end{CODEBOX}

initialize-function �ϥ��ץꥱ������󤬵�ư�����Ȥ��Τ��ʴ����Ѵ��ξ��֤�
���ꤷ�ޤ���
����������ꥷ�����󥹤���ʬ�ˤϵ�ǽ̾�򵭽Ҥ��ޤ���
ʣ���ε�ǽ̾����ꤹ�뤳�Ȥ��ǽ�Ǥ���

���Ȥ��С����ץꥱ�������ץ������ε�ư�������ܸ����ϥ⡼�ɤǤ��ä�
�ۤ������ϡ�

\begin{CODEBOX}
(initialize-function '(japanese-mode)) \\
\end{CODEBOX}

�Τ褦�˵��Ҥ��ޤ���

��ư���ˤϥ���ե��٥åȥ⡼�ɤǤ��äƤۤ�������\XFER �����ܸ����ϥ⡼�ɤ�
�ܹԤ����Ȥ��˥������ʥ١����Υ⡼�ɤǤ��äƤۤ������ϼ��Τ褦�˵��Ҥ��ޤ���

\begin{CODEBOX}
(initialize-function '(japanese-mode base-katakana \\
�������������������������������������������� alpha-mode)) \\
\end{CODEBOX}

% ------------------------------ (10)
\hptar{label=H0.4.6.4.0.10}
\item {\dg\bf �ۤ���\HIDX{�������ޥ����ե�������ɤ߹���}{�������ޤ����դ�
����Τ�ߤ���}{H0.4.6.4.0} $-$ \HIDX{load}{�̣ϣ���}{H0.4.6.4.0}}

\vspace{5mm}

\begin{CODEBOX}
(load "�ե�����̾") \\
\end{CODEBOX}

.canna ����ۤ��Υ������ޥ����ե�������ɤ߹��ळ�Ȥ��Ǥ��ޤ����ե�����̾
�����Хѥ�̾�ǽ񤯤��Ȥ�˾�ޤ����Ǥ����ۡ���ǥ��쥯�ȥ��ɽ��������ޡ�
���ϻ��ѤǤ��ޤ���

% ------------------------------ (11)
\hptar{label=H0.4.6.4.0.11}
\item {\dg \HIDX{�С�������ɽ���ѿ�}{�ϡ������򤢤�魯�ؤ󤹤�}
{H0.4.6.4.0}}

�ʲ����ѿ��ˤϡؤ���ʡ٤ΥС�����󡢤��ʴ����Ѵ������ФΥС������
���ʴ����Ѵ������ФȤ��̿����Ѥ����Ƥ���ץ��ȥ���ΥС������
���줾���Ǽ����Ƥ��ޤ���

\begin{CODEBOX}
\verb+ canna-version     ����    �ؤ���ʡ٤ΥС������ + \\
\verb+ protocol-version  ����    �ץ��ȥ���ΥС������ + \\
\verb+ server-version    ����    �����ФΥС������ + \\
\end{CODEBOX}

���Ȥ��Сؤ���ʡ�Version3.2 �Ǥ� canna-version �ˤ� 3002 �����äƤ��ޤ���
���Τ褦�����Ѥ��뤳�Ȥ��Ǥ��ޤ���

\vspace{5mm}

{\tt
\begin{center}
\begin{tabular}{|p{1cm}p{12cm}|}
\hline
(��) & \\
 & (if (> protocol-version 1999) (setq auto t)) \\
 & ���ʴ����Ѵ��ץ��ȥ��뤬2.0�ʾ�ΤȤ��˸¤ä��༡��ư�Ѵ������Ѥ��ޤ��� \\
\hline
\end{tabular}
\end{center}
}

\vspace{5mm}

% ------------------------------ (12)
\hptar{label=H0.4.6.4.0.100}
\item {\dg ���Τۤ��Υ������}
\label{�������ޥ���������¾�Υ������}

�ʲ��Ǥ��Τۤ��Υ�����ɤˤĤ����������ޤ���
% misao �� �ѿ��Ǥʤ���Τ⤢��褦�ʤΤǡ������ȥ����Ȥ��롣
%
%�����Υ�����ɤϤ��٤��ѿ��Ǥ��Τǡ�setq ���Ѥ����ͤ����ꤹ�뤳�Ȥˤ�ä�
%�������ޥ������Ƥ���������
%
%\begin{nquote}{2em}
%\begin{verbatim}
%(��)
%    ;�����޻������Ѵ��ơ��֥�Ȥ��Ƽ�ʬ�κ���������Τ��Ѥ��ޤ���
%    (setq romkana-table "myromaji.kp")
%\end{verbatim}
%\end{nquote}

\begin{itemize}
% ------- ��
\hptar{label=H0.4.6.4.0.101}
\item \HIDX{english-table}{�ţΣǣ̣ɣӣȡݣԣ��£̣�}{H0.4.6.4.0} 

������Ѵ��ơ��֥��̾������ꤷ�ޤ���������Ѵ��ơ��֥�ϥ���
�޻������Ѵ��ơ��֥��Ʊ�������Ǥ���

������Ѵ��ơ��֥�Υ������ѥ���ͥ���̤ϥ����޻������Ѵ��ơ��֥�
�Υ������ѥ���Ʊ���Ǥ���

\begin{nquote}{2em}
\begin{verbatim}
(��) 
    ;������Ѵ��ơ��֥�Ȥ��� english.kp ���Ѥ��ޤ���
    (setq english-table "english.kp")
\end{verbatim}
\end{nquote}

% ------- ��
\hptar{label=H0.4.6.4.0.102}
\item \HIDX{cursor-wrap}{�ãգңӣϣҡݣףң���}{H0.4.6.4.0}

�ɤߤ����Ϥ��Ƥ�����֤�����ɽ�����Ƥ�����֤ǥ���������ư����Ȥ���
��ü���鱦�ذ�ư�������򤷤��Ȥ��亸ü���麸���ư�������򤷤��Ȥ���ȿ
��¦��ü�˥������뤬��ư���뤳�Ȥ���ꤷ�ޤ���t �ǰ�ư����nil �ǰ�ư����
���󡣥ǥե���Ȥ� t �Ǥ���

\begin{nquote}{2em}
\begin{verbatim}
(��) 
    ;���������åפ��ʤ��褦�ˤ��ޤ��� 
    (setq cursor-wrap nil) 
\end{verbatim}
\end{nquote}

% ------- ��
\hptar{label=H0.4.6.4.0.103}
\item \HIDX{numerical-key-select}{�Σգͣţңɣã��̡ݣˣţ١ݣӣţ̣�
�ã�}{H0.4.6.4.0}

���������ɽ�����Ƥ���Ȥ��ˡ������������Ѥ��Ƹ��������Ǥ��뤫�ɤ�����
���ꤷ�ޤ���t ���뤤�� nil ����ꤷ�ޤ����ǥե���Ȥ� t �Ǥ���nil ����ꤹ��
�ȡ����������򲡤���������򤵤�Ƥ�����䤬���ꤷ��������������
�����ɤߤ����ϤȤ��Ƽ�갷���ޤ���

\begin{nquote}{2em}
\begin{verbatim}
(��)
    ;�����������������ǹԤ��ޤ���
    (setq numerical-key-select nil) 
\end{verbatim}
\end{nquote}

% ------- ��
\hptar{label=H0.4.6.4.0.104}
\item \HIDX{select-direct}{�ӣţ̣ţãԡݣģɣңţã�}{H0.4.6.4.0}

numerical-key-select �� t �Ǥ���Ȥ��ˡ����������Ǹ�������򤷤��Ȥ���
�������ɽ���Τޤޤ������Ǥʤ�������ꤷ�ޤ���
t �ξ��ϸ����������λ���ޤ����ǥե���Ȥ� nil �Ǥ���

\begin{nquote}{2em}
\begin{verbatim}
(��) 
    ;�����������Ϥǰ����⡼�ɤ�λ���ޤ���
    (setq select-direct t)
\end{verbatim}
\end{nquote}

% ------- ��
\hptar{label=H0.4.6.4.0.105}
\item \HIDX{bunsetsu-kugiri}{�£գΣӣţԣӣաݣˣգǣɣң�}
{H0.4.6.4.0}

�����ɽ�����Ƥ���Ȥ���ʸ�ᤴ�Ȥ˶���Ƕ��ڤ뤫�ɤ�������ꤷ�ޤ���
t �Ƕ��ڤ�ޤ����ǥե���Ȥ� nil �Ǥ���

\begin{nquote}{2em}
\begin{verbatim}
(��) 
    ;����ɽ�����֤�ʸ����ڤ��Ԥ��ޤ���
    (setq bunsetsu-kugiri t)
\end{verbatim}
\end{nquote}

% ------- ��
\hptar{label=H0.4.6.4.0.106}
\item \HIDX{character-based-move}{�ãȣ��ң��ãԣţҡݣ£��ӣţġݣͣ�
�֣�}{H0.4.6.4.0}

�ɤߤ����Ϥ��Ƥ���Ȥ��˥��������ư��Ԥ����ˡ�ʸ��ñ�̤ǰ�ư��Ԥ���
�ɤ�������ꤷ�ޤ����ǥե���Ȥ� t �Ǥ���character-based-move �� nil �ˤ���
�ȡ������޻������Ѵ��γ����ñ�̤��Ȥˤ��ƥ���������ư���ޤ�������
�����"ju" �����Ϥ������ϡ��ؤ���٤ϰ�ʸ���Ȥߤʤ���ƥ������뤬��
ư���ޤ���ʸ������ΤȤ���Ʊ�ͤ˼�갷���ޤ���

\begin{nquote}{2em}
\begin{verbatim}
(��) 
    ;�ɤߤ����Ϥ��Ƥ���Ȥ������������ʸ��ñ�̤ǰ�ư���ޤ���
    (setq character-based-move nil)
\end{verbatim}
\end{nquote}

% ------- ��
\hptar{label=H0.4.6.4.0.107}
\item \HIDX{reverse-widely}{�ңţ֣ţңӣšݣףɣģţ̣�}
{H0.4.6.4.0}

t ����ꤹ����ɤߤ����Ϥ��Ƥ���Ȥ���ʸ�����ȿž�ϰϤ������ʤ�ޤ�����
�ե���Ȥ� nil �Ǥ���

\begin{nquote}{2em}
\begin{verbatim}
(��) 
    ;�ɤ����ϻ����ɤ����Τ�ȿž�����ޤ���
    (setq reverse-widely t)
\end{verbatim}
\end{nquote}

% ------- ��
\hptar{label=H0.4.6.4.0.108}
\item \HIDX{quit-if-end-of-ichiran}{�ѣգɣԡݣɣơݣţΣġݣϣơݣɣ�
�ȣɣң���}{H0.4.6.4.0}

�������ɽ���򤷤Ƥ���Ȥ��ˡ��ǽ������ɽ�����Ƥ�����֤Ǽ���������Ԥ���
�������ɽ����λ�����ɤߤ��Τ�Τ����Ȥ���ɽ������褦�ˤʤ�ޤ���
2 ���ܤΥ��ڡ���������������ɽ���˥������ޥ������Ƥ���Ȥ�
�ʤɤ� quit-if-end-of-ichiran �� t �ˤ��Ƥ����������Ǥ����ǥե���Ȥ� nil �Ǥ���

\begin{nquote}{2em}
\begin{verbatim}
(��) 
    (setq quit-if-end-of-ichiran t) 
\end{verbatim}
\end{nquote}

% ------- ��
\hptar{label=H0.4.6.4.0.109}
\item \HIDX{break-into-roman}{�£ңţ��ˡݣɣΣԣϡݣңϣͣ���}{H0.4.6.4.0}

�Хå����ڡ����������Ǥä��Ȥ���ľ���Υ����޻������Ѵ����줿ʸ����
�����޻�����뤫�ɤ�������ꤷ�ޤ����ǥե���Ȥ� nil �Ǥ���

\begin{nquote}{2em}
\begin{verbatim}
(��) 
    (setq break-into-roman t) 
\end{verbatim}
\end{nquote}

% ------- ��
\hptar{label=H0.4.6.4.0.110}
\item \HIDX{gakushu}{�ǣ��ˣգӣȣ�}{H0.4.6.4.0}

���ʴ����Ѵ����ؽ���Ԥ����ɤ�������ꤷ�ޤ����ǥե���Ȥ� t �Ǥ���

\begin{nquote}{2em}
\begin{verbatim}
(��)
    (setq gakushu t)
\end{verbatim}
\end{nquote}

% ------- ��
\hptar{label=H0.4.6.4.0.111}
\item \HIDX{stay-after-validate}{�ӣԣ��١ݣ��ƣԣţҡݣ֣��̣ɣģ���
��}{H0.4.6.4.0}

�������ɽ�����֤Ǹ�������򤷤�ñ�����ɽ�����֤���ä��Ȥ��ˡ�
������ʸ��򼡤�ʸ��˰�ư�����뤫�ɤ�������ꤷ�ޤ���
nil �ǥ�����ʸ��ϼ���ʸ��˰�ư���ޤ���
t �Ǥϥ�����ʸ����Ѥ��ޤ��󡣥ǥե���Ȥ� t �Ǥ���

\begin{nquote}{2em}
\begin{verbatim}
(��)
    ;�������򤹤�ȼ���ʸ��˰�ư���ޤ��� 
    (setq stay-after-validate nil)
\end{verbatim}
\end{nquote}

% ------- ��
\hptar{label=H0.4.6.4.0.112}
\item \HIDX{kakutei-if-end-of-bunsetsu}{�ˣ��ˣգԣţɡݣɣơݣţΣġ�
�ϣơݣ£գΣӣţԣӣ�}{H0.4.6.4.0}

�DZ�ʸ��Ǽ�ʸ��ذ�ư���褦�Ȥ����Ȥ��ˡ����ꤹ�뤫�ݤ�����ꤷ�ޤ���
t �dz��ꤷ�ޤ���nil ����ꤹ��ȺǺ�ʸ�ᤫ������ʸ��ˤʤ�ޤ���
�ǥե���Ȥ� nil �Ǥ����ޤ���stay-after-validate �� nil ��
�Ȥ��� kakutei-if-end-bunsetsu �� t �ΤȤ��ϺDZ�ʸ��Ǹ���������֤�������
���򤹤����ʸ�����ꤷ�ޤ���

\begin{nquote}{2em}
\begin{verbatim}
(��)
    (setq kakutei-if-end-of-bunsetsu nil)
\end{verbatim}
\end{nquote}

% ------- ��
\hptar{label=H0.4.6.4.0.113}
\item \HIDX{grammatical-question}{�ǣң��ͣͣ��ԣɣã��̡ݣѣգţӣԣ�
�ϣ�}{H0.4.6.4.0}

ñ����Ͽ���ʻ����ꤷ���塢�ܺ٤��ʻ�ʬ��Τ���μ����Ԥ����ݤ���
���ꤷ�ޤ���t �Ǽ����Ԥ���nil �Ǥϼ���򤷤ޤ��󡣥ǥե���Ȥ� t �Ǥ���

\begin{nquote}{2em}
\begin{verbatim}
(��)
    (setq grammatical-question nil)
\end{verbatim}
\end{nquote}

% ------- ��
\hptar{label=H0.4.6.4.0.114}
\item \HIDX{n-henkan-for-ichiran}{�ΡݣȣţΣˣ��Ρݣƣϣҡݣɣãȣɣ�
����}{H0.4.6.4.0}

�Ѵ�����(�ǥե���ȤǤϥ��ڡ���������Xfer)�򲿲󤫲����ȸ������ɽ����
�Ԥ���褦�ˤ��뤳�Ȥ��Ǥ��ޤ���n-henkan-for-ichiran ���Ф��ƿ�����ꤹ��ȡ�
���ꤵ�줿����Ѵ������򲡤����Ȥˤ����������ɽ������ޤ���
0 ����ꤹ��Ȳ����Ѵ������򲡤��Ƥ����ɽ�����Ԥ��ʤ��ʤ�ޤ���
�ǥե���Ȥ� 2 �ǡ�2 ���Ѵ������򲡤��Ȱ���ɽ���ˤʤ�ޤ���

\begin{nquote}{2em}
\begin{verbatim}
(��)
    ;�Ѵ������� 3 �󲡤��ȸ��������ɽ������ޤ���
    (setq n-henkan-for-ichiran 3)
\end{verbatim}
\end{nquote}

% ------- ��
\hptar{label=H0.4.6.4.0.115}
\item \HIDX{kouho-count}{�ˣϣգȣϡݣãϣգΣ�}{H0.4.6.4.0}

��������ʤɰ���ɽ���򤷤Ƥ���Ȥ��ˡ��������򤷤Ƥ�����ܤ�
���Τι��ܤβ����ܤǤ���Τ���ɽ�����뤫�ɤ�������ꤷ�ޤ���
t ��ɽ������nil ��ɽ�����ޤ���
�ǥե���Ȥ� t �Ǥ���

\begin{nquote}{2em}
\begin{verbatim}
(��)
    (setq kouho-count 3)
\end{verbatim}
\end{nquote}

% ------- ��
\hptar{label=H0.4.6.4.0.116}
\item \HIDX{auto}{���գԣ�}{H0.4.6.4.0}

�༡��ư�Ѵ����Ѥ��뤫�ɤ�������ꤷ�ޤ���t ���༡��ư�Ѵ����Ѥ��ޤ���
�ǥե���Ȥ� nil �Ǥ���

\begin{nquote}{2em}
\begin{verbatim}
(��)
    (setq auto nil)
\end{verbatim}
\end{nquote}

% ------- ��
\hptar{label=H0.4.6.4.0.117}
\item \HIDX{n-kouho-bunsetsu}{�Ρݣˣϣգȣϡݣ£գΣӣţԣӣ�}{H0.4.6.4.0}

�༡��ư�Ѵ����˴����˼�ưŪ���Ѵ����줿ʸ��򤤤��Ĥޤdz��ꤻ�����ݻ����뤫��
���ꤷ�ޤ���3 �� 32 ���ϰϤǻ��ꤷ�Ƥ���������
�ǥե���Ȥ� 16 �Ǥ���

\begin{nquote}{2em}
\begin{verbatim}
(��)
    (setq n-kouho-bunsetsu 16)
\end{verbatim}
\end{nquote}

% ------- ��
\hptar{label=H0.4.6.4.0.118}
\item \HIDX{abandon-illegal-phonogram}{���£��ΣģϣΡݣɣ̣̣ţǣ���
�ݣУȣϣΣϣǣң���}{H0.4.6.4.0}

�����޻������Ѵ��������ʥ����޻������ϤȤ��ƻĤ뤫�ɤ�������ꤷ�ޤ���
�ǥե���ȤǤ������ʥ����޻��ϻĤ�ޤ�����abandon-illegal-phonogram �� t ��
���ꤹ��������ʥ����޻������Ϥ���ΤƤ��ޤ���
�ǥե���Ȥ� nil �Ǥ���

\begin{nquote}{2em}
\begin{verbatim}
(��)
    (setq abandon-illegal-phonogram nil)
\end{verbatim}
\end{nquote}

% ------- ��
\hptar{label=H0.4.6.4.0.119}
\item \HIDX{hex-direct}{�ȣţءݣģɣңţã�}{H0.4.6.4.0}

16 �ʥ��������ϻ��� 4 �����ܤ����줿������ 16 �ʥ��������ϥ⡼�ɤ���λ���뤫
�ɤ�������ꤷ�ޤ���
nil ����ꤹ��� 4 �����ܤ����줿�����Ǥ����Ϥ����ꤷ�ʤ����֤Ȥʤ�ޤ���
�ǥե���Ȥ� nil �Ǥ���

\begin{nquote}{2em}
\begin{verbatim}
(��)
    (setq hex-direct nil)
\end{verbatim}
\end{nquote}

% ------- ��
\item \HIDX{index-hankaku}{�ɣΣģţءݣȣ��Σˣ��ˣ�}{H0.4.6.4.0}

����������ֹ������ʸ������Ⱦ��ʸ�����ѹ����뤫�ɤ�������ꤷ�ޤ���

�ǥե���Ȥ� nil(����ʸ��)�Ǥ���

% ------- ��
\hptar{label=H0.4.6.4.0.120}
\item \HIDX{index-separator}{�ɣΣģţءݣȣ��Σˣ��ˣ�}{H0.4.6.4.0}

����������ֹ��Ⱦ��ʸ���˥������ޥ������Ƥ�����Ρ��ֹ�ȸ���Ȥζ�
�ڤ�Υ��ѥ졼����ʸ������ꤷ�ޤ���? �μ��ΰ�ʸ���򥻥ѥ졼����ʸ����
���ƻ��Ѥ��ޤ���
�ǥե���Ȥϥԥꥪ�ɤǤ���

\begin{nquote}{2em}
\begin{verbatim}
(��)
    ;���ѥ졼�����򥳥���ˤ��ޤ���
    (setq index-separator ?:)
\end{verbatim}
\end{nquote}

% ------- ��
\hptar{label=H0.4.6.4.0.121}
\item \HIDX{n-keys-to-disconnect}{�Ρݣˣţ٣ӡݣԣϡݣģɣӣãϣΣΣ�
�ã�}{H0.4.6.4.0}

����ե��٥åȤ����Ϥ�³�������Υ����ФȤ���³���ڤ�륹�ȥ���������
���ꤷ�ޤ���
�ǥե���Ȥ� 500 �Ǥ���0 ����ꤷ�����ϡ������ФȤ���³���ڤ�ʤ��ʤ�
�ޤ���

\begin{nquote}{2em}
\begin{verbatim}
(��)
    (setq n-keys-to-dicsonnect 500)
\end{verbatim}
\end{nquote}

% ------- ��
\hptar{label=H0.4.6.4.0.122}
\item \HIDX{allow-next-input}{���̣̣ϣסݣΣţأԡݣɣΣУգ�}{H0.4.6.4.0}

�������ɽ�����֤ǿ����ʳ��Υ����򲡤����Ȥ��ˡ����ߥ������뤬���֤��Ƥ�
����䤬���򤵤켡�����ϤȤʤ뤫�ɤ�������ꤷ�ޤ���
allow-next-input �� t �ξ�硢�������ɽ�����֤ǿ����ʳ��Υ������ǤĤȡ�����
���ϤȤʤ�ޤ���
allow-next-input �� nil �ξ�硢�֥ԥáפȤ��������Ĥꡢ�������Ϥˤ��Ѥ�
���ޤ���
�ǥե���Ȥ� t �Ǥ���

\begin{nquote}{2em}
\begin{verbatim}
(��)
    (setq allow-next-input t)
\end{verbatim}
\end{nquote}

% ------- ��
\hptar{label=H0.4.6.4.0.123}
\item \HIDX{ignore-case}{�ɣǣΣϣңšݣã��ӣ�}{H0.4.6.4.0}

�����޻������Ѵ��ơ��֥�ˤ����̾�ʸ���ε�§�������������Ƥ��ꡢ�̾�
����ʸ�������Ϥ���ȥ���ե��٥åȤΤޤ����Ϥ��졢���ʤˤ��Ѵ�����ޤ�
��
ignore-case �� t �ˤ�����硢��ʸ�������Ϥ��Ƥ�����޻������Ѵ��ǤϾ�ʸ��
�Ȥ��Ƽ�갷���ޤ���
�ǥե���Ȥ� nil �Ǥ���

\begin{nquote}{2em}
\begin{verbatim}
(��)
    (setq ignore-case nil)
\end{verbatim}
\end{nquote}

% ------- ��
\hptar{label=H0.4.6.4.0.124}
\item \HIDX{define-esc-sequence}{�ģţƣɣΣšݣţӣáݣӣţѣգţΣã�}
{H0.4.6.4.0}

ü���Υ����פ��Ȥ˥��������ץ������󥹤ȵ�ǽ�������б������ꤷ�ޤ���
vi(1) �� cannad(1) �ʤɤ�
�ե��󥯥���󥭡��ʤɤ��᤹��ݤ˻��Ȥ���ޤ���

�ʤ������ץꥱ�������ˤ�äƤϤ��ε�ǽ���б����Ƥ��ʤ����Ȥ�����ޤ���

\begin{nquote}{2em}
\begin{verbatim}
(��)
    (define-esc-sequence "kterm" "\ESC[28~" ?\Help)
    (define-esc-sequence "kterm" "\ESC[210z" ?\Xfer)
\end{verbatim}
\end{nquote}

�����ߥʥ�̾�ȥ��������ץ������󥹤�ʸ���������ޤ������Ǹ�Υ�����ñ�����Ȥ��뤿�ᡢʸ������ꤹ��褦�˹�ʸ��������ä��Ƥ��ޤ���

% ------- ��
\hptar{label=H0.4.6.4.0.125}
\item \HIDX{quickly-escape-from-kigo-input}
{�ѣգɣãˣ̣١ݣţӣã��Уšݣƣңϣ͡ݣˣɣǣϡݣɣΣУգ�}{H0.4.6.4.0}

�������ϥ⡼�ɻ��ˡ������Ϣ³�������Ϥ��뤫�ɤ�������ꤷ�ޤ���
quickly-escape-from-kigo-input �� t �ξ�硢�������ϥ⡼�ɤǵ���� 1 ��
���Ϥ���ȵ������ϥ⡼�ɤ�λ���ޤ���
quickly-escape-from-kigo-input �� nil �ξ�硢��������򤷤��Τ���
�������ϥ⡼�ɤˤȤɤޤꡢ�����Ϣ³�������Ϥ��뤳�Ȥ��Ǥ��ޤ���
�ǥե���Ȥ� nil �Ǥ���

\begin{nquote}{2em}
\begin{verbatim}
(��)
    ;�����Ϣ³�������Ϥ��ޤ���
    (setq quickly-escape-from-kigo-input nil)
\end{verbatim}
\end{nquote}

% ------- ��
\hptar{label=H0.4.6.4.0.126}
\item \HIDX{henkan-or-self-insert}{�ȣţΣˣ��Ρݣϣҡݣӣţ̣ơݣɣΣӣţң�}
{H0.4.6.4.0}

Ϣʸ���Ѵ����Ѥ��Ƥ����硢�١������ؤ���ǽ�����Ѥ��ƥ١��������ؤ����ݤ�
������ư�����ꤷ�ޤ���

���Ȥ��С��١������ؤ���ǽ�����Ѥ��ơ�bese-eisu ��ǽ�ˤ��
�ѿ����ϥ⡼�ɤ˰ܹԤ���ݡ����ϥ١������Ҥ餬�ʤξ����Ѵ���Ԥ���
�Ҥ餬�ʰʳ��ξ��� self-insert ��Ԥ�����ˤϰʲ��Τ褦�˵��Ҥ��ޤ���

\begin{nquote}{2em}
\begin{verbatim}
(��)
    ;�ɤ�������� C-o �򲡤����Ȥˤ��
    ;���ܸ����ϥ⡼�ɤ� ON/OFF ��Ԥ���
    (set-key 'yomi-mode "\C-o" 'base-kana-eisu-toggle)
    ;�١������Ҥ餬�ʰʳ��ξ��� self-insert ��Ԥ���
    (set-key 'yomi-mode "\Space" 'henkan-or-self-insert)
\end{verbatim}
\end{nquote}

% ------- ��
\hptar{label=H0.4.6.4.0.127}
\item \HIDX{henkan-or-do-nothing}{�ȣţΣˣ��ΡݣϣҡݣģϡݣΣϣԣȣɣΣ�}
{H0.4.6.4.0}

�༡��ư�Ѵ����Ѥ��Ƥ����硢�١������ؤ���ǽ�����Ѥ��ƥ١��������ؤ����ݤ�
������ư�����ꤷ�ޤ���

���Ȥ��С����������ϻ����Ѵ���Ԥ��褦�˻��ꤷ�Ƥ��뤬���١�����
�Ҥ餬�ʰʳ��ξ����Ѵ���Ԥ�ʤ��褦�ˤ��뤿��ˤϰʲ��Τ褦�˵��Ҥ��ޤ���

\begin{nquote}{2em}
\begin{verbatim}
(��)
    ;�������ϻ����Ѵ���Ԥ������١�����
    ;�Ҥ餬�ʰʳ��ξ��Ϥʤˤ�Ԥ�ʤ���
    (set-key 'chikuji-yomi-mode "." 
      '(self-insert henkan-or-do-nothing))
    (set-key 'chikuji-bunsetsu-mode "." 
      '(self-insert henkan-or-do-nothing))
\end{verbatim}
\end{nquote}
    
% ------- ��
\hptar{label=H0.4.6.4.0.128}
\item \HIDX{romaji-yuusen}{�ңϣͣ��ʣɡݣ٣գգӣţ�}{H0.4.6.4.0}

romaji-yuusen �� t �ξ�硢���ߤ����ϥ⡼�ɤ��ɤ����ϥ⡼�ɤޤ���
����˽ऺ��⡼�ɤǤ��ꡢ�����޻������Ѵ�������ξ��֤ΤȤ���
��������ʸ���������޻������Ѵ���ͭ���ʥ����Ǥ���С�self-insert ���Ԥ��ޤ���
romaji-yuusen �� nil �ξ�硢�ɤΤ褦�ʾ��֤ˤ����Ƥ⡢�����˳�����Ƥ�줿
��ǽ���¹Ԥ���ޤ���
�ǥե���Ȥ� nil �Ǥ���

\begin{nquote}{2em}
\begin{verbatim}
(��)
    ;�����޻������Ѵ���ͥ�褷�ޤ���
    (setq romaji-yuusen t)
\end{verbatim}
\end{nquote}

% ------- ��
\hptar{label=H0.4.6.4.0.129}
\item \HIDX{auto-sync}{���գԣϡݣӣ٣Σ�}{H0.4.6.4.0}

ñ����Ͽ�������ľ��˼�ưŪ�˼���ν񤭽Ф�������Ԥ����ɤ�������ꤷ�ޤ���
auto-sync �� t �ξ��ϡ���ưŪ�˼���ν񤭽Ф�������Ԥ��ޤ���
auto-sync �� nil �ξ��ϡ�������Ф��륢���������ʤ��ʤä�������
����ν񤭽Ф��������Ԥ��ޤ���
�ǥե���Ȥ� t �Ǥ���

\begin{nquote}{2em}
\begin{verbatim}
(��)
    ;ñ����Ͽ�����ľ��˼���ν񤭽Ф�������Ԥ��ޤ���
    (setq auto-sync t)
\end{verbatim}
\end{nquote}


\end{itemize}

\end{enumerate}


\subsection{���Τۤ���\HIDX{Lisp�δؿ�}{�̣ɣӣФΤ��󤹤�}{H0.4.6.5.0}}

���ޤ����������ʳ��� Lisp �δؿ����������ޤ���
�ʲ��δؿ��Ϥ��ʴ����Ѵ��ˤ�ľ�ܤϱƶ���ڤܤ��ޤ���

\begin{enumerate}

% ------------------------------ (1)
\item {\dg\bf ���ʬ�� $-$ cond, if}

\begin{nquote}{3em}
\begin{verbatim}
(if A B C)
\end{verbatim}
\end{nquote}

���μ��ϡ��ޤ� A ��ɾ���������η�̤� nil �ʳ��Ǥ���С�B ��ɾ������
nil �Ǥ���� C ��ɾ�����ޤ���
(if�ġ�) ���η�̤� B �ޤ��� C ��ɾ����̤Ȥʤ�ޤ���

\begin{CODEBOX}
(��) \\
����(if (> protocol-version 2000) (setq auto t) \\
����������������������������������������(setq auto nil)) \\
����(setq auto (if (> protocol-version 2000) t nil)) \\
�� \\
\end{CODEBOX}

\begin{nquote}{3em}
\tt
(cond (C$1$ B$1$) (C$2$ B$2$)��������(C$n$ B$n$))
\end{nquote}

�ޤ��� C$1$ ��ɾ���������η�̤� nil �ʳ��ʤ� B$1$ ��ɾ�������η�̤� 
cond ���η�̤Ȥ��ޤ������ξ�� C$2$ �� C$n$��B$2$ �� B$n$ ��ɾ����
��ޤ��� C$1$ ��ɾ����̤� nil �ʤ鼡�� C$2$ ��ɾ�����ޤ��� C$2$ ��
��̤� nil �ʳ��ʤ� B$2$ ��ɾ�������η�̤� cond ���η�̤Ȥ��ޤ���

C$i$ �򼡡��� nil �ʳ����ͤ��ФƤ���ޤ�ɾ������ nil �ʳ����ͤ��֤ä�
�褿�Ȥ����� B$i$ ��ɾ����������� cond �����ͤȤ����֤��ޤ���

% ------------------------------ (2)
\item {\dg\bf ���Τۤ������湽¤ $-$ let}

\begin{nquote}{3em}
\tt
(let ((VAR$1$ VAL$1$) (VAR$2$ VAL$2$) .... (VAR$n$ VAL$n$)) body)
\end{nquote}

�ѿ� VAR$1$ �� VAR$n$ �ˡ����줾���� VAL$1$ �� VAL$n$ ������������� 
body ��ɾ�����ޤ��� body ��ɾ����̤� let ���η�̤Ȥʤ�ޤ��� body ��
ʣ���μ��򵭽Ҥ��뤳�Ȥ���ǽ�Ǥ��� let ����ɾ������λ����� VAR$1$ �� 
VAR$n$ ���ͤϸ����ͤ����ޤ���

% ------------------------------ (3)
\item {\dg\bf �ؿ����༡�ƽФ� $-$ progn}

\begin{nquote}{3em}
\begin{verbatim}
(progn A B C ....)
\end{verbatim}
\end{nquote}

�� A B C �� �򼡡���ɾ�����ƺǸ�μ���ɾ����̤��ͤȤ��ޤ���
if ʸ�μ¹�����ʣ���ν����򵭽Ҥ���Ȥ��˻Ȥ��ޤ���

\begin{CODEBOX}
(��) \\
����(if (< canna-version 2000) \\
������(progn (setq n-henkan-for-ichiran 2) \\
��������������(setq quit-if-end-of-ichiran t)) ) \\
\end{CODEBOX}

% ------------------------------ (4)
\item {\dg\bf �ͤ���� $-$ equal, =, ��, ��, eq }

�ͤ���Ӥ�Ԥ��ޤ���eq �ϥ���ܥ�Ʊ�Τ���Ӥȿ���Ʊ�Τ���ӤΤߤˤ���
�Ѥ��뤳�Ȥ��Ǥ��ޤ��󤬡����˹�®����ӽ�����Ԥ��ޤ���

% ------------------------------ (5)
\item {\dg\bf �����ͤ�Ĵ�� $-$ not, and, or}

�����ͤ�Ĵ�����ޤ���

and ����� or �ϰ�����¦����ɬ�פʿ�����ɾ�����ޤ���
���ʤ����and �ξ��Ϻǽ�� nil ���̤Ȥ����֤��������ä���硢
�������鱦�˽񤫤�Ƥ��������ɾ�����ޤ���
or �ξ��ϡ��ǽ�� nil �ʳ����ͤ��̤Ȥ����֤��������ä���硢
�������鱦�˽񤫤�Ƥ��������ɾ�����ޤ���

% ------------------------------ (6)
\item {\dg\bf ���ѱ黻 $-$ $+$, $-$, $\ast$, /, \%}

���ѱ黻��Ԥ��ޤ�����̣�� C ����ξ���Ʊ���Ǥ���

% ------------------------------ (7)
\item {\dg\bf �����٥����쥯����� $-$ gc}

Lisp ����Υ�����������ƥ��ư���������������Ԥ��ޤ���

% ------------------------------ (8)
\item {\dg\bf �ꥹ�ȱ黻 $-$ list, cons, car, cdr, atom, null}

�ꥹ�Ƚ�����Ԥ��ޤ������ʴ����Ѵ����Ѥ��뤳�ȤϤʤ��Ǥ��礦��
�ܺ٤� Lisp ��Ϣ�λ��ͽ��������������

% ------------------------------ (9)
\item {\dg\bf ���� $-$ quote}

\begin{nquote}{3em}
\begin{verbatim}
(quote A)
\end{verbatim}
\end{nquote}

'A ��Ʊ���褦�� A �Ȥ����ǡ����򤽤Τޤޥǡ����Ȥ��ƻȤ������Ȥ����Ѥ��ޤ���

% ------------------------------ (10)
\item {\dg\bf ���� $-$ set}

set �� setq ��Ʊ�ͤν�����Ԥ��ޤ���setq �Ȥΰ㤤���� 1 ������ɾ���������Ǥ���

% ------------------------------ (11)
\item {\dg\bf �ؿ���� $-$ defun, defmacro}

�ؿ�������ޥ��������Ԥ��ޤ������ʴ����Ѵ����Ѥ��뤳�ȤϤʤ��Ǥ��礦��
�ܺ٤� Lisp ��Ϣ�λ��ͽ��������������

% ------------------------------ (12)
\item {\bf\dg ����ܥ��ʣ�� $-$ copy-symbol}

����ܥ��ʣ����������ޤ���
copy-symbol �ϥ���ܥ�λ��Ĥ��ޤ��ޤ�°���򤽤Τޤޥ��ԡ����ޤ���

\begin{CODEBOX}
(��) \\
����(copy-symbol '�ɤߥ⡼�� 'yomi-mode) \\
\end{CODEBOX}

�嵭�ν����ˤ�� yomi-mode �Ȥ���������ɤ�����ˡ��ɤߥ⡼�ɡפ�
���Ҥ��뤳�Ȥ��Ǥ���褦�ˤʤ�ޤ���

\end{enumerate}

% ------------------------------ 3.3.7
\section{��������˥������ޥ������Ƥߤ���}
\HIDXAS{�������ޥ�������}{�������ޤ����Τ줤}{H0.4.7.0.0}
\label{3.3.7��������˥������ޥ������Ƥߤ���}

�ʲ��Ǥϡ����Τ褦�˥������ޥ�������Ȥ��Υ������ޥ����ե������
������򼨤��ޤ���

\vspace{5mm}

%{\tt
\begin{center}
\begin{tabular}{|l|p{12cm}|}
\hline
\hpref{count=1,label=H0.4.7.0.0.1}1 & �Ѵ��� Xfer �ǹԤ������ڡ��������Ѷ���ʸ�������ϤȤ��ƻ��Ѥ������� \\
\hpref{count=1,label=H0.4.7.0.0.2}2 & �������������������������ȴ�������� \\
\hpref{count=1,label=H0.4.7.0.0.3}3 & ��������򤷤��鼡��ʸ��˰�ư�������� \\
\hpref{count=1,label=H0.4.7.0.0.4}4 & ����Ϻ\footnotemark �Τ褦�����ˡ�����Ϥ������� \\
\hpref{count=1,label=H0.4.7.0.0.5}5 & �֤ȡΤ�ξ���Ȥ������� \\
\hpref{count=1,label=H0.4.7.0.0.6}6 & ��ĥ��˥塼��ʬ������������� \\
\hline
\end{tabular}
\footnotetext{����Ϻ�ϥ��㥹�ȥ����ƥ�(��)�ξ�ɸ�Ǥ�}
\end{center}
%}

\vspace{5mm}

�ʲ��Υ������ޥ����������Ǥϡ�.canna ��񤭴����뤳�Ȥˤ�ä�
�������ޥ�����¸����ޤ����񤭴����ϡ�.canna �κǸ�����˲ä��Ƥ�����
�ɤ��Ǥ��礦��.canna ���ʤ����ϡ����Υ��ޥ�ɤ����Ϥ��ơ�.canna ��
�������Ƥ���������

\begin{CODEBOX}
\% cp \refCANNALIBDIR /sample/default.canna .canna \\
\% chmod u+w .canna \\
\end{CODEBOX}

\begin{enumerate}

% ------------------- (1)
\item {\dg\bf �Ѵ��� Xfer �ǹԤ������ڡ��������Ѷ���ʸ�������ϤȤ��ƻ��Ѥ�����}
\hptar{label=H0.4.7.0.0.1}

�ޤ������ڡ����������Ф����̾�����ϵ�ǽ(self-insert)�������Ƥޤ���
self-insert �� self �Ȥϥ������Ȥ�ؤ��ޤ���
���ʤ�������ڡ����������Ф��� self-insert �������Ƥ��
�������������Ȥ�����ǽ�����ڡ��������˳�����Ƥ��ޤ���

���ڡ��������� self-insert ��ǽ�������Ƥ�ˤϼ��Τ褦��
�������ޥ����ե����뤷�ޤ���

\begin{CODEBOX}
(global-set-key "\verb!\!Space" 'self-insert) \\
\end{CODEBOX}

"$\backslash$Space" �ϥ��ڡ��������򼨤��ޤ���
"$\backslash$Space" ������� "  " �Τ褦�˵��Ҥ��Ƥ�Ʊ����̣�ˤʤ�ޤ���

���ơ�����������ꤹ��ȥ��ڡ����������Ѵ����˻��ѤǤ��ʤ��ʤꡢ
��������Ѷ������������褦�ˤʤ�ޤ���

���Ѷ���ǤϤʤ���Ⱦ�Ѷ���ˤ������Ȥ���
�����޻������Ѵ��ơ��֥�ν񤭴�����ɬ�פȤʤ�ޤ���

% ------------------- (2)
\item {\dg �������������������������ȴ������}
\hptar{label=H0.4.7.0.0.2}

�ؤ���ʡ٤Ǥϡ����������ɽ�����Ƥ�����֤dzƸ�����ֹ�򲡤�����硢
�������뤬���θ���˰�ư��������Ǽºݤ�����ϹԤ��ޤ���

\begin{SCREEN}
{\tt \%} \underline{��������}\CURSOR \\
{[ �� ]} \\
\end{SCREEN}

%\begin{center} ���Ѵ� \end{center}
���������Ѵ�

\begin{SCREEN}
\verb+%+ \fbox{���} \\
{[ �� ]} \\
\end{SCREEN}

%\begin{center}   ���Ѵ�  \end{center}
���������Ѵ�

\begin{SCREEN}
\verb+%+ \fbox{����} \\
{[ �� ]  1 ���  \fbox{2} ����  3 ����  4 �߼�  5 ����} \SCREENRIGHT{2/8}\\
\end{SCREEN}

%\begin{center}     ��4������  \end{center}
��������4������

\begin{SCREEN}
\verb+%+ \fbox{�߼�} \\
{[ �� ]  1 ���  2 ����  3 ����  \fbox{4} �߼�  5 ����} \SCREENRIGHT{4/8}\\
\end{SCREEN}

�ºݤ����򤹤뤿��ˤϳ��ꥭ��(�ǥե���ȤǤ� \RETURN)�򲡤��ʤ���Фʤ�ޤ���

���������򤷤������Ǥ��θ�������򤷡��������ɽ����λ����������
���Τ褦�˥������ޥ����ե����뤷�ޤ���

\begin{CODEBOX}
(setq select-direct t) \\
\end{CODEBOX}

����ˤ�ꡢ���������򤷤������Ǹ��䤬���򤵤졢�����������λ���ޤ���

\begin{SCREEN}
\verb+%+  \fbox{����} \\
{[ �� ]  1 ���  \fbox{2} ����  3 ����  4 �߼�  5 ����} \SCREENRIGHT{2/8}\\
\end{SCREEN}

%\begin{center} ��4������   \end{center}
��������4������

\begin{SCREEN}
\verb+%+ \fbox{�߼�} \\
{[ �� ]} \\
\end{SCREEN}

% ------------------- (3)
\item {\dg ��������򤷤��鼡��ʸ��˰�ư������}
\hptar{label=H0.4.7.0.0.3}

Ĺ��ʸ�Ϥ����Ϥ����Ѵ������򲡤�������ʸ�Ϥ����������򤹤�Ȥ�����ˡ��
���Ϥ��Ƥ�������¿���Ǥ��礦��
���Τ褦�����ξ�硢������������򤷤��Ȥ��˥�����ʸ�᤬Ʊ��ʸ���
�ȤɤޤäƤ��ޤ��Ȥ�������ʸ���ư�򤷤ʤ���Фʤ�ʤ��Τ����ؤ�
�����ƤϤ��ʤ��Ǥ��礦����

\begin{SCREEN}
\setlength{\fboxsep}{\reversefboxsep}
\verb+%+ \fbox{���ˤ�}\underline{���ۤ��褦�ȹͤ��Ƥ��ޤ����ɤ����ޤ��礦��}  \\
{[ �� ]  1 ���Ҥ� \fbox{2} ���ˤ�  3 �߼֤�  4 ���Ԥ�} \SCREENRIGHT{2/8} \\
\end{SCREEN}

%\begin{center} ������(\RETURN)������ \end{center}
������������(\RETURN)������

\begin{SCREEN}
\setlength{\fboxsep}{\reversefboxsep}
\verb+%+ \fbox{���ˤ�}\underline{���ۤ��褦�ȹͤ��Ƥ��ޤ����ɤ����ޤ��礦��}  \\
{[ �� ]} \\
\end{SCREEN}

�ְ��������򤷤��Τ����餽��ʸ��ǤλŻ��Ͻ���äƤ���Ϥ�����
��������򽪤�ä��鼡��ʸ��˰�ư���Ƥۤ������פȤ������ˤ��פ�������
¿���Ǥ��礦�����Τ褦�ʾ��ϡ����Τ褦�˥������ޥ������ޤ���

\begin{CODEBOX}
(setq stay-after-validate nil) \\
\end{CODEBOX}

\begin{SCREEN}
\setlength{\fboxsep}{\reversefboxsep}
\verb+%+ \fbox{���ˤ�}\underline{���ۤ��褦�ȹͤ��Ƥ��ޤ����ɤ����ޤ��礦��} \\
{[ �� ]  1 ���Ҥ� \fbox{2} ���ˤ�  3 �߼֤�  4 ���Ԥ�} \SCREENRIGHT{2/8} \\
\end{SCREEN}

%\begin{center} ������(\RETURN)������ \end{center}
������������(\RETURN)������

\begin{SCREEN}
\setlength{\fboxsep}{\reversefboxsep}
\verb+%+ \underline{���ˤ�}\fbox{���ۤ��褦��}\underline{�ͤ��Ƥ��ޤ����ɤ����ޤ��礦��} \\
{[ �� ]} \\
\end{SCREEN}

% ------------------- (4)
\item {\dg ����Ϻ\footnotemark �Τ褦�����ˡ�����Ϥ�����}
\hptar{label=H0.4.7.0.0.4}

\footnotetext{����Ϻ�ϥ��㥹�ȥ����ƥ�(��)�ξ�ɸ�Ǥ�}

�ְ���Ϻ�˴���Ƥ���Τǰ���Ϻ�ȤǤ������Ʊ���������ܸ����Ϥ���������
�Ȥ������⤤��ä����Ǥ��礦��

����Ϻ�Ȥۤ�Ʊ�����������󶡤��륫�����ޥ����ե����뤬\\
\refCANNALIBDIR /sample/just.canna�Ȥ����󶡤���Ƥ��ޤ��Τǡ��ʲ��Υ��ޥ�ɤ�
�¹Ԥ��ơ����Υե�����򥫥����ޥ����ե�����Ȥ��ƻ��Ѥ���褦�ˤ��Ƥ���������

\begin{CODEBOX}
\% cd \\
\% cp \refCANNALIBDIR /sample/just.canna .canna \\
\end{CODEBOX}

% ------------------- (5)
\item {\dg �֤ȡΤ�ξ���Ȥ�����}
\hptar{label=H0.4.7.0.0.5}

��ץ��ˤ�äƤϡ��֡���(�ԥꥪ��)���Ǥä��Ȥ��ˡ֡��פ����Ϥ���뤫
�֡��פ����Ϥ���뤫������Ǥ����ꡢ
��[\ ��(�Ѥ��ä�)�����Ϥ����Ȥ��ˡ֡�\ ��(�������ä�)�����Ϥ���뤫
�֡�\ ��(���ѳѤ��ä�)�����Ϥ���뤫������Ǥ����ꤹ���Τ�����ޤ���

�ؤ���ʡ٤Ǥ�����޻������Ѵ��ơ��֥��񤭴����뤳�Ȥˤ�ꡢ
�֡��פ�֡�\ �פ��б����뵭����ѹ����뤳�Ȥ��Ǥ��ޤ�����
�����޻������Ѵ��ơ��֥�ν񤭴����ϼ㴳���ݤǤ⤢��ޤ���

���Τ褦�ʾ�硢�ʲ��Τ褦�˥������ޥ������뤳�Ȥˤ�äơ�
���줾��ε����ξ��������Ƥ�����
ɬ�פ˱����ƴ�ñ���ڤ��ؤ��뤳�Ȥ��Ǥ��ޤ���

\begin{CODEBOX}
\verb!(defsymbol ?. "��" "." )! \\
\verb!(defsymbol ?[ "��" "[" )! \\
\end{CODEBOX}

���Τ褦�ˤ��Ƥ����Ȱʲ��μ��ǵ�����ڤ��ؤ��뤳�Ȥ��Ǥ��ޤ���

�ޤ����֡��פ����Ϥ��ޤ���

\begin{SCREEN}
\underline{��} \\
{[ �� ]} \\
\end{SCREEN}

���������Ѵ������򲡤��ޤ���


\begin{SCREEN}
\setlength{\fboxsep}{\reversefboxsep}
\underline{��}  \\
{[ �� ] ��\fbox{1} �� 2 ��} \\
\end{SCREEN}

����������\fbox{��}��\RETURN �� 2 �����Ӥޤ���

\begin{SCREEN}
\underline{��} \\
{[ �� ]} \\
\end{SCREEN}

�Ĥ�����ϥԥꥪ�ɤ��ǤĤȡ֡��פ�����ޤ���

\begin{SCREEN}
\underline{��} \\
{[ �� ]} \\
\end{SCREEN}

Ʊ�ͤ����ˤ�긵���᤹���Ȥ���ǽ�Ǥ���

% ----------------------- (6)
\item {\dg ��ĥ��˥塼��ʬ�����������}
\hptar{label=H0.4.7.0.0.6}

�ǥե���ȤǤϡ�\HELP �����򲡤����Ȥ��ˡ������ɥ饤���
�桼�ƥ���ƥ���˥塼��ɽ�����졢�������ϡ�ñ����Ͽ���Ѵ��������ѹ��ʤɤ�
�Ԥ��ޤ�����������������ˡ�Ǥϡ���ʬ���Ȥ�������ǽ�򤹤������򤹤뤳�Ȥ�
�Ǥ��ʤ��ä��ꡢ�������ϻ��˼�ʬ�����򤷤�������򤹤��˸��Ĥ��뤳�Ȥ�
�Ǥ��ʤ��ä��ꤹ�뤳�Ȥ�����ޤ���

���Τ褦�ʾ�硢��˥塼��ʬ�ǥ������ޥ������뤳�Ȥ��Ǥ��ޤ���

�����Ǥϡ���ʬ��ʸ�������������������ˡ���ʬ�λȤ�
�䤹���褦�˥�˥塼��������Ƥߤޤ��礦��

���Ȥ��С�

\begin{nquote}{2em}
\begin{enumerate}
\item �Ѵ��������ѹ��ʤɤϤۤȤ�ɹԤ�ʤ��Τǡ���˥塼�ˤ�
����ɽ����ñ����Ͽ����������Ф�����

\item ����ɽ���Ǥϡ���ʬ�Ǻ���������̤ȿ��ص���ΰ�����ɽ����������
\end{enumerate}
\end{nquote}

�Ȥ������ˤϡ��ʲ��Τ褦�˥������ޥ������ޤ���

%�ʲ��ˡ���˥塼������ȡ�ʸ�������κ�����Ʊ���˹Ԥä���򼨤��ޤ���

\vspace{5mm}
%{\tt 
\begin{center}
\begin{tabular}{|ll|}
\hline
\multicolumn{2}{|l|}{\verb+; ʸ������ kakko �����+} \\
\multicolumn{2}{|l|}{\verb+(defselection kakko "[���]" '(?�� ?�� ?�� ?�� ?�� ?�� ?�� ?�� ?�� ?��))+} \\
\multicolumn{2}{|l|}{\verb+; ʸ������ keisan �����+} \\
\multicolumn{2}{|l|}{\verb+(defselection keisan "[�׻�]"+} \\
\multicolumn{2}{|l|}{\verb+��'(?�� ?�� ?�� ?�� ?�� ?�� ?�� ?�� ?�� ?�� ?�� ?�� ?��))+} \\
                                    & \\
\multicolumn{2}{|l|}{\verb+; ��˥塼�˰�������Ͽ���������+} \\
\verb+(defmenu  toplevel-menu+      & \verb+; ��� menu �����+ \\
\verb+  ("����" menu1)+             & \verb+; menu1 ��+ \\
\verb+  ("��Ͽ" menu2)+             & \verb+; menu2 ��+ \\
\verb+)+                            & \\
                                    & \\
\multicolumn{2}{|l|}{\verb+; ���������򤹤�ȡ���̡����ص��桢���ꥷ���ΰ���ɽ�����Ǥ���+} \\
\verb+(defmenu menu1+               & \verb+; menu1 �����+ \\
\verb+  ("���å�" kakko)+           & \verb+; kakko �����ɽ������+ \\
\verb+  ("�׻�" keisan)+            & \verb+; keisan �����ɽ������+ \\
\verb+  ("���ꥷ��" greek-mode)+    & \verb+; ��ǽ greek-mode ��Ԥ�+ \\
\verb+)+                            & \\
                                    & \\
\multicolumn{2}{|l|}{\verb+; ��Ͽ�����򤹤�ȡ�ñ����Ͽ/��������������ɽ�����Ǥ���+} \\
\verb+(defmenu menu2+               & \verb+; menu2 �����+ \\
\verb+  ("��Ͽ" touroku-mode)+      & \verb+; ��ǽ touroku-mode ��Ԥ�+ \\
\verb+  ("���" delete-dic-mode)+   & \verb+; ��ǽ delete-dic-mode ��Ԥ�+ \\
\verb+  ("�������" jisho-ichiran)+ & \verb+; ��ǽ jisho-ichiran ��Ԥ�+ \\
\verb+)+                            & \\
                                    & \\
\multicolumn{2}{|l|}{\verb+; C-t �򲡤��ȡ����������˥塼��ɽ������+} \\
\multicolumn{2}{|l|}{\verb+(set-key 'empty-mode "+$\backslash$\verb+C-t" 'toplevel-menu)+} \\
\hline
\end{tabular}
\end{center}
%}
\vspace{5mm}

���Τ褦�˥������ޥ�������ȡ��ʲ��Τ褦�˥�˥塼�����ѤǤ��ޤ���

\begin{enumerate}

\item \CTRL + \fbox{t} �򲡤��ơ���˥塼�����롣
\label{initmenu}

\begin{SCREEN}
\verb+% + \CURSOR \\
{[��ĥ] \fbox{1} ����\,\,\, 2 ��Ͽ} \SCREENRIGHT{1/2} \\
\end{SCREEN}

\item {\dg\bf 1 ����} �����򤹤롣
\label{slctichiran}

\begin{SCREEN}
\verb+% + \CURSOR \\
{[��ĥ] \fbox{1} ���å�\,\,\, 2 �׻�\,\,\, 3 ���ꥷ��} \SCREENRIGHT{1/3} \\
\end{SCREEN}

\item {\dg\bf 1 ���å�} �����򤹤롣

\begin{SCREEN}
\verb+% + \CURSOR \\
{[���] \fbox{��}\,\,\, ��\,\,\, ��\,\,\, ��\,\,\, ��\,\,\, 
��\,\,\, ��\,\,\, ��\,\,\, ��\,\,\, ��} \SCREENRIGHT{1/10} \\
\end{SCREEN}

\item (\ref{slctichiran})�ǡ�{\dg\bf 2 �׻�} �����򤹤롣

\begin{SCREEN}
\verb+% + \CURSOR \\
{[�׻�] \fbox{��}\,\,\, ��\,\,\, ��\,\,\, ��\,\,\, ��\,\,\, ��\,\,\, ��\,\,\, 
��\,\,\, ��\,\,\, ��\,\,\, ��\,\,\, ��\,\,\, ��} \SCREENRIGHT{1/13} \\
\end{SCREEN}

\item (\ref{initmenu})�ǡ�{\dg\bf 2 ��Ͽ} �����򤹤롣

\begin{SCREEN}
\verb+% + \CURSOR \\
{[��ĥ] \fbox{1} ��Ͽ\,\,\, 2 ���\,\,\, 3 �������} \SCREENRIGHT{1/3} \\
\end{SCREEN}

\end{enumerate}

\end{enumerate}




\section{Robots}
\subsection{The Killer Robots from Hell}

The robots strategy is based on the following variables, and the code that
interprets them.

\subsection{Speeds}
\begin{description}
\end{description}


\section{Hints}
\subsection{X Customization}
\subsection{Credits, Bug Reports, etc}
\subsection{Key Table}

\end{document}
