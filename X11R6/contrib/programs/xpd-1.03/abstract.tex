\voffset -1 in
\documentstyle{report}
\pagestyle{empty}
\begin{document}
\vsize = 11 in
\font\bold cmbx10
\font\bigbold cmbx10 scaled \magstep 1
\font\biggbold cmbx10 scaled \magstep 2
\font\bigggbold cmbx10 scaled \magstep 4
\font\reg cmr10 scaled \magstep 2

\centerline{\bigggbold XPD}
\bigskip
\centerline{\biggbold A Process Manager for the X Window System}
\medskip
\centerline{\bigbold By:  Vincent M. Tkac}
\centerline{\bold vmtkac@acm.org}
\bigskip
{\bold CS890 \hfill Dr. Paul M. Mullins}
\bigskip\bigskip

\leftline{\bold Abstract:}
\medskip

XPD is designed to provide an easy-to-use, point-and-click interface to a correlation of process information available through {\bold ps}.
XPD does not use {\bold ps}, it reads the kernel and creates an image of the active process table, giving the user a ``snapshot'' of process activity.

XPD utilizes the Athena Widget set and was written under SunOS 4.1 on a Sun X.
Code that may be specific to the Sun architecture is located in a separate object file.

\medskip
\bigskip
\leftline{\bold Program Features:}
\begin{enumerate}
\item{Display all process users}
\item{Display processes owned by a user}
\item{Display trace of process ancestry and children}
\item{Display full command line arguments for a process}
\item{Signal (kill) a selected process}
\item{Specify the desired signal to be sent to the selected process}
\item{Signal (kill -9) all of a users processes}
\item{Rescan the process table from the kernel}
\item{Display list of defunct processes}
\item{Display number of processes}
\item{Display detailed information for a selected process}
\begin{enumerate}
\item{Effective uid, login name and name of process owner}
\item{Parent process id}
\item{Process group id}
\item{Process status}
\item{tty}
\end{enumerate}
\end{enumerate}

\medskip
\leftline{\bold Note:}
\medskip

XPD is intended to be run with an effective uid of root.
Any user in the {\bold wheel} group is given permission to kill any process.
Users not in the operator group are only allowed to kill processes mathcing their effective uid.

\end{document}
