\documentstyle[11pt, titlepage]{article}

\parindent0pt

\begin{document}


\title{{\Huge TeXcad}\\
       {\large Version 1.3 \\
               {\sc FreeWare}}}
               
\author{Dokumentation}

\maketitle

\section{Preface}

     This is TeXcad Version 1.3\footnote{This document dscribes xtexcad 1.2}
     
  
     TeXcad is distributed under the concept of FreeWare.\\ \\
     Copyright (c) 1991 xtexcad  V1.3 by K. Zitzmann

     The X Consortium, and any party obtaining a copy of these files from
     the X Consortium, directly or indirectly, is granted, free of charge, a
     full and unrestricted irrevocable, world-wide, paid up, royalty-free,
     nonexclusive right and license to deal in this software and
     documentation files (the "Software"), including without limitation the
     rights to use, copy, modify, merge, publish, distribute, sublicense,
     and/or sell copies of the Software, and to permit persons who receive
     copies from any such party to do so.  This license includes without
     limitation a license to do the foregoing actions under any patents of
     the party supplying this software to the X Consortium.\\ \\

     The author makes no guarantees with this program and
     is not responsible for any damage caused by it.
     
     

\section{Introduction}

\LaTeX~is a professional '{\it document preparation system}' for all
kinds of documents like books, articles etc.
It is obvious that those documents can contain graphics.
Although \LaTeX~offers some graphics primitives, it is 
difficult and awkward to produce a picture with them.
The TeXcad package is a comfortable editor for \LaTeX~ pictures
which translates the drawn images into pure \LaTeX~ code and also
allows a subsequent editing.\\
\newline

  \hfill    Klaus Zitzmann,\hfill Koblenz/Germany in Oktober 91 \\
  
\newpage
  
\section{User Guide}

\subsection{Installation}

The program is written in ANSI-C and compiled with the\\
GNU-C-Compiler.
It uses the MIT's X-Window-System and includes the major
part of the ATHENA Widget Set. It runs under UNIX.
I've developed and tested this program on a SUN 3 and 4 (Sparc Sun).\\
\\
The underlying version of TeXcad is distributed with the files given in the README file. Please check these file before installing this package.\\


If you use the {\it standardmakefile}, type 'make -f stdmake cad' to produce the
code named 'xtexcad'.\\
Type {\it xrdb -load Xtexcad.ad} to load the resources.
You can start the program by typing 'xtexcad [filename] \{toolkit options\}'\\
or 'texcad [filename] \{toolkit options\}'.
If you use the Imakefile just type 'xmkmf' to
generate a makefile from the imakefile. Then type 'make all' to produce the code\footnote{or better type 'make install' if you have got the permission !}.

\subsection{Particularities}

The biggest picture the present program normally creates is of A4 size.\\
If this is not large enough, you have to use the {\bf move image} option {\it (see below)}
or just modify the '{\it unitlength-parameter}' within your TeX file\footnote{Manipulating 
the {\it unitlength} parameter always leads to a complete recomputation
of circle diameters. Since these diameters are limited, it is possible that the results of this
recomputation are invalid diameters, which are not displayed in the output.\\
Nevertheless it is possible to abuse this modification as a {\it n-level zoom}.}.
The internal unit of measurement is points (pt). All relevant
values of an external picture will be transformed while reading
it into the database.

During the development of TeXcad, I tried to keep the code as
portable as possible\footnote{TeXcad was successfully installed on NeXT and Data General
computers at the University of Koblenz.}.
Nevertheless, there occurred multiple 
problems, especially when using the {\it NeWS} Server with the {\it twm} Window Manager.



\subsection{Commands}

TeXcad supports the following commands:

\begin{itemize}
  \item line 
  \item vector
  \item box (normal, framed, filled)
  \item circle (normal,oval,filled) 
  \item b\'{e}zier
  \item text
\end{itemize}


     {\bf {\it CONVENTION:}}
\\
     Pressing button 1 on an icon selects the function.
     Watch the status line of the window manager. The current
     context and possible user actions are shown there.
     When moving the pointer into the {\bf canvas}\footnote{The area in which the picture is drawn.}
     area, the selected command becomes active. Move the pointer out of
     the canvas to cancel the current command.


Sometimes the computer displays certain messages in a
popup window.
If there is no other possibility, you can close this window
by moving the pointer (in and) out of this window.


\begin{description}
  \item{\bf The line and vector functions:} \\
     The start and end points of a line are specified by pressing
     button 1 on the canvas.
     Press one of the other buttons to start drawing a new line.
  \item{\bf The box function:} \\
     Hold button 1 on the box icon to pop up a menu.
     Move the pointer on one of the options ({\bf framed, filled,
     dashed}). Release the button to select the desired function.
     After the rectangle has been drawn, you are prompted for the
     box text and its position.

  \item{\bf The circle function:} \\
     (see {\it function box} !)

  \item{\bf The b\'{e}zier function:} \\
     Draws a quadratic B-Spline curve.
     After determining start and end points of the curve ({\it see line/vector}) 
     you have to control its curvature.
     Be sure to add the b\'{e}zier style option to the header of your document
     (e.g.~{\it documentstyle[11pt, bezier]\{book\}}).
     
  \item{\bf The text function:} \\
     After pressing button 1 on the canvas, you are prompted for
     textual input.
     You can specify the text position relative to the 
     selected point.\\
     Examples:\\
     Press BOTTOM to center the text relative to this point.\\
     Press TOP RIGHT and this point will be the upper right edge
     of the text.
\end{description}



The following commands allow a convenient editing of the
picture:


\begin{description}
  \item{\bf zoom:} \\
     A constant zoom area is supported.
     There is only one zoom level possible (and necessary !).
  \item{\bf refresh:} \\
     Redraws the screen.
     Leaves the zoom function.
  \item{\bf erase:} \\
     Holding button 1 on the 'erase' icon pops up a menu
     with the options {\bf erase entire database} and {\bf pick erase}.\\
     The first option deletes all objects.\\
     With the second option, you can erase individual
     objects from a complex picture by picking them with
     the pointer: move the pointer close to the desired object
     and press button 1. The object will be marked on the
     screen and you can delete it by pressing the second button.
     Sometimes, when editing complex pictures where objects
     are very close to each other or overlapping, it can occur
     that the computer picks the wrong object. In this case,
     use button 3 to select the next possible object in the
     vicinity.
  \item{\bf copy:} \\
     Copies the object.\\
     First you have to pick the desired object ( see {\it erase} !).
     Hold button 2 and move the pointer to position the object
     on the screen.
     
    
  \item{\bf pickedit:} \\
     Holding button 1 on the erase icon pops up a menu
     with the option {\bf pick object} and some {\bf TEXT-edit} options.

     What concerns the first option:\\
          Pick the desired object!
          To move the object, just hold button 2 and move the
          pointer.
          The object gets markers (little circles). Move the
          pointer into one of these markers and hold button 1 to
          edit the object. Only circle objects don't get
          markers. It is sufficient to move the pointer close to
          the arc.\\

     The other options:\\
          These options ({\bf TEXT, framedBox Text, dashedBox Text})
          allow a comfortable editing of text.

  \item{\bf Settings:} \\
     It is possible to switch between different default settings within
     the `settings` menu:
     \begin{itemize}
        \item Unlimited Slopes \\
          \LaTeX~offers only limited vector and line slopes.
          This option turns this restriction off but be sure
          that your TeX Version can handle these slopes.
        \item Watch Line Length \\
          The minimum length of a line or a vector is 10 points.
          By turning this option on, the computer will not allow to
          draw a shorter line/vector.

        \item Enable Raster \\
          Shows a raster.
        \item Snap Pointer \\
          If the raster is active, the pointer will be snapped
          on the\\
          knot-points of the raster.
        \item Cross Wire \\
          Creates an additional pointer - a cross wire.

        \item Center DIN A4 \\
          The computer suggests that the picture shall be
          centered horizontally and vertically on the screen.
          The final adjustment can be done 'by hand' only!
          Some hints are generated within the *.tex file.

        \item Automatic Refresh \\
          Some floating point operations cause an internal
          rounding error so that sometimes individual objects
          cannot be deleted correctly.
          The refresh removes this garbage automatically.
        \item Enable Ruler\\
          Shows a ruler.
          The unit of measurement is pt (points).
     \end{itemize}


  \item{\bf The function file:} \\
     This function manages the i/o of pictures.
     The file format is pure \LaTeX~code.
     The structure of a file, created by TeXcad, is the
     following:
     \begin{verbatim}
          %
          % /* comments */
          %
          begin{picture}(xdim,ydim)(xref,yref)
               .
               .
               /* LaTeX graphic commands */
               .
               .
          end{picture}
     \end{verbatim}

     While reading an external document into the database,
     TeXcad skips all unknown commands until it recognizes the
     picture environment. This environment will be extracted
     from the document.

  \item{\bf Coordinates: } \\
     The absolute and relative (relative to the zoom area) coordinates
     are displayed at the top right corner of the screen.
     Sometimes this is useful for a correct positioning of the
     pointer.

  \item{\bf Move picture: } \\
     Select one of the four arrows diplayed below the coordinates box to move the
     entire picture to the corresponding direction. Selecting the square will center
     your picture.
     
\end{description}

\section{Credits}

	There are many people who were involved in producing TeXcad, among them are Randolf Werner who wrote the file-select box, 
	Ulrich Koch who developed  the scanner definition for lex, Karl Heinz Staudt who wrote the basic routines of module file\_sel.c, Juergen Marenda
	and Fritz Haubensak. I want to thank them all for their invaluable support.
	
\section{Bibliography}

	Just read all manuals on X.
	
	

\end{document}
 
