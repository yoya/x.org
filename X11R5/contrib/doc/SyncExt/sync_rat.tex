% Copyright 1991 by Olivetti Research Limited, Cambridge, England and
% Digital Equipment Corporation, Maynard, Massachusetts.
%
%                        All Rights Reserved
%
% Permission to use, copy, modify, and distribute this software and its 
% documentation for any purpose and without fee is hereby granted, 
% provided that the above copyright notice appear in all copies and that
% both that copyright notice and this permission notice appear in 
% supporting documentation, and that the names of Digital or Olivetti
% not be used in advertising or publicity pertaining to distribution of the
% software without specific, written prior permission.  
%
% DIGITAL AND OLIVETTI DISCLAIM ALL WARRANTIES WITH REGARD TO THIS SOFTWARE,
% INCLUDING ALL IMPLIED WARRANTIES OF MERCHANTABILITY AND FITNESS, IN NO EVENT
% SHALL THEY BE LIABLE FOR ANY SPECIAL, INDIRECT OR CONSEQUENTIAL DAMAGES OR
% ANY DAMAGES WHATSOEVER RESULTING FROM LOSS OF USE, DATA OR PROFITS, WHETHER
% IN AN ACTION OF CONTRACT, NEGLIGENCE OR OTHER TORTUOUS ACTION, ARISING OUT
% OF OR IN CONNECTION WITH THE USE OR PERFORMANCE OF THIS SOFTWARE.

\documentstyle{article}

\setlength{\parindent}{0 pt}
\setlength{\parskip}{6pt}

\begin{document}

\begin{center}

{\large X SYNCHRONIZATION EXTENSION}\\[10pt]
{\large Overview of Version 2.0}\\[15pt]
{\it Tim Glauert}\\[0pt]
{\tt thg@cam-orl.co.uk}\\[0pt]
{\bf Olivetti Research/MultiWorks}\\[5pt]
{\it Jim Gettys}\\[0pt]
{\tt jg@crl.dec.com}\\[0pt]
{\bf Digital Equipment Corporation,}\\[0pt]
{\bf Cambridge Research Laboratory}

\end {center}

\subsection*{Motivation}

The motivation behind this extension came from work with multi-media
applications, particularly those involving digital video and animation. Such
applications perform reasonably well when run on an unloaded local machine or
over an unloaded network, but synchronization cannot be maintained in a world
wide distributed systems environment, with the attendant delays caused by an
unpredictable network, the speed of light and paging on the client machine.
By providing a mechanism whereby the server can internally coordinate
when requests are executed, applications can work much more closely to real
time.  More sophisticated multi-media applications require synchronization of
graphics between two or more clients which can greatly increase the problems.

\subsection*{Animation}

Applications which use animation need to display an image or some graphics at
regular intervals. Without this extension, such applications must keep track
of real-time internally, and deliver requests to X at the time they are
needed. This relies on there being a consistent delay between the generation
of the requests by the client and the execution of the requests by the server,
and in most cases such applications also rely on this delay being short in
relation to the time between frames. It is common for such clients to
synchronize with the server after each frame to ensure that the server is
processing the requests within the required time.

Using the synchronization extension it is possible to send a number of frames
of animation to an X server as a single unit. The server then interprets the
synchronization requests between the frames and plays the animation at the
correct speed. This provides two major benefits:

\begin{description}

\item[Batching requests]

The synchronization is done within the server and so it is possible to send
the requests for a number of frames in a single batch.  This can make the data
transport to the server more efficient, particularly if the client is on a
different machine from the server and the data must travel over a network.  In
the case where the client is on the local machine, batching can reduce the
number of process-switches required, and more importantly, the number
of network packets transported.

\item[Low Latency]

The fact that the synchronization is done in the server can reduce the latency
between the time a frame is supposed to be displayed and the time at which the
server processes the requests for that frame. The delay caused by the data
transport from the client to the server is removed, although there may still
be a delay if the server is busy processing another client's requests.

\end{description}

All servers which implement the synchronization extension provide a {\bf
Counter} called SERVERTIME which counts up in milliseconds from some arbitrary
starting point. A client can {\bf Query} the value of this counter and can
send an {\bf Await} request which will block until this counter reaches some
later value. When that value is reached, the client is unblocked and
processing continues with the request following the {\bf Await}. An example
animation application would send the following requests to display 5 frames of
animation at 40 millisecond intervals.

\begin{center}
\begin{tabular}{l}
	{\bf Await} 1000\\
	{\bf PutImage} {\it image1}\\
	{\bf Await} 1040\\
	{\bf PutImage} {\it image2}\\
	{\bf Await} 1080\\
	{\bf PutImage} {\it image3}\\
	{\bf Await} 1120\\
	{\bf PutImage} {\it image4}\\
	{\bf Await} 1160\\
	{\bf PutImage} {\it image5}\\
\end{tabular}
\end{center}

Applications can easily ask a server to do more than it possibly can execute
in the time available.  Yet only the application can determine how to shed
load to the server to keep near real time.  We examined mechanisms by which
requests already buffered in the connection might be ignored. In the end, we
agreed that the complications that would ensue in such semantics were not
worth it, and that notification of the application was both necessary and
sufficient. Thus the extension will generate an event when the server cannot
meet the specified timing requirements, so that the client can take some
action to reduce the load. The client may, for example, skip a few frames of
animation or reduce the detail in an image.

This is achieved by specifying a {\it threshold} with an await request. If the
counter has already exceeded the wait value by at least this threshold, an
event is generated. This informs the client that the server is failing to keep
up with the requests it is being given or, less likely, that the client is
sending the requests too late.

\subsection*{Other system counters}

Some implementations of this extension may provide other system counters in
addition to SERVERTIME. One very useful system counter is FRAMEEND, which
advances by 1 when the display hardware has finished displaying a new frame.
By preceding some graphics requests with a wait on FRAMEEND, a client can
ensure that each frame of the animation corresponds to a single frame on the
display. In addition, an efficient server may be able to process these
graphics requests when the screen is not being updated by the hardware, thus
reducing or eliminating flicker.

The extension implementation allows other extensions to define their own
counters. A video extension may provide a counter which advances in step with
some incoming analogue video signal. The multi-buffering extension may provide
a counter which can be used to synchronize buffer-switching between different
multi-buffering clients.

\subsection*{Multiple clients}

As well as using the system-provided counters, clients can
create their own counters for inter-client synchronization. The value of
such counters is changed by an {\bf Advance} request which adds a small number
(usually 1) to the current counter value.

Graphics overlays can be implemented using counters, for example. One client
is responsible for generating the background image and this client uses a
system counter as before. It also has its own counter which it advances after
the background has been drawn. A second client is responsible for the overlay,
and it waits for the first client's counter to be advanced before drawing its
graphics. This ensures that the background is always drawn before the overlay
without the need for the two clients to communicate directly with each other
at every frame.

By using a pair of counters it is possible to implement a rendezvous between
two clients, so that, for example, there could be a pair of animations where
are kept synchronized to each other but not to real time. Although it would be
possible to make the clients follow one another as described above, the
rendezvous scheme has the advantage that the requests for the two clients can
be executed in parallel.

There is a danger of deadlock in some of the more complicated synchronization
systems, but this can be avoided by using an await request which waits for one
of a set of counters to change. A client can wait on SERVERTIME as well as on
another counter so that the wait will time-out if the other client does not
respond.

One mechanism which is absent from this extension is mutual-exclusion.
Although a number of schemes for mutual exclusion have been proposed, none of
them are entirely satisfactory. More importantly, there are no obvious
example applications which require {\it fine-grain} mutual-exclusion within
the server. Coarse-grain mutual exclusion can be implemented using the
selection mechanism or {\bf GrabServer/UngrabServer} requests.

\subsection*{Alarms}

This extension also provides {\bf Alarms}, which allow a client to have an
event delivered when a counter reaches a certain value. This is most often
used in conjunction with the SERVERTIME counter, or another real-time counter,
to have events delivered at regular real-time intervals.  An alarm is
associated with a counter and has a {\it value} at which it will generate the
next event. When the counter reaches this value, an event is generated and the
value is increased by some {\it increment}. This process continues until the
alarm reaches a {\it limit} value, at which point a last event is generated
and the alarm becomes {\bf Exhausted}.

Alarms may also be used to keep track of the progress of other synchronized
activity by selecting for events from a counter created by one of a set of
co-operating clients.

\subsection*{Availability}

The original design was implemented over a year ago; this revised
specification was prepared in reaction to experience in its use. A sample
implementation using the X11R5 server of this revised specification is
currently close to completion, and will be made available by the end of
November 1991.

% ???, once conversion of example programs and implementation
% FRAMEEND system counters are complete.

This extension will be proposed as an X Consortium Standard and, if accepted,
it will be available in the next release of X.  For more details on this
extension email the authors.

\subsection*{Changes from version 1.0}

The protocol description for version 1.0 of this extension was published as
part of the proceedings of the 5th Annual X Technical Conference. The major
differences between that version and version 2.0 are:

\begin{itemize}

\item Alarms have been added.

\item System counters have been added.

\item {\bf Timers} have been replaced by the SERVERTIME system counter.

\item The {\bf AutoAdvance} mechanism, which allowed a counter to be advanced
when a given event was delivered, has been removed. The system counter
mechanism provides similar functionality in a simpler and more efficient
manner.

\item Counters are now 64-bit entities, which removes the need to handle the
problem of counter wrap-around.

\end{itemize}

\subsection*{Acknowledgements}

The idea of blocking clients within the server came from Todd Brunhoff of
Tektronix, and the basic elements of this extension were developed by Tim
Glauert. Dave Carver suggested the Alarms mechanism and provided a first
implementation of it. Jim Gettys made many useful comments on the
specification and Bob Scheifler made some small but crucial observations, as
he does in most things related to X.

\end{document}
